Existence of \emph{interpolant} formulas was proved by W. Craig \cite{CraigA} for classical predicate logic. In this section, we prove \emph{Craig's interpolation} theorem for $\ldl$ using a method similar to Maehara \cite{maehara1960interpolation}.

\begin{thm}[Craig's Interpolation for $\ldl$]\label{thm:craig} For any $\Gamma_1$, $\Gamma_2$ and $\Delta$ in the language $\L$, if $\LDL \vdash \Gamma_1 , \Gamma_2 \Rightarrow \Delta$, then there is a formula $C$ in the language $\L$ such that
  \begin{enumerate}
    \item $\LDL \vdash \Gamma_1 \Rightarrow C$,
    \item $\LDL \vdash \Gamma_2 , C \Rightarrow \Delta$, and
    \item $V(C) \subseteq V(\Gamma_1) \cap V(\Gamma_2 , \Delta)$.
  \end{enumerate}
\end{thm}

\begin{proof}
Let $\D$ be a proof for $\Gamma_1 , \Gamma_2 \Rightarrow \Delta$ in $\LDL$. By induction on the height of $\D$, we construct an interpolant which satisfies (1), (2) and (3), in different cases for the last rule of $\D$. In cases for the rules $L \wedge ^n$, $L \vee ^n$, $L \supset ^n$ and $L \rightarrow ^n$, we also need to consider whether the principal formula is in $\Gamma_1$ or $\Gamma_2$ in separate cases.

From the induction hypothesis, we can assume for any smaller proof-tree which proves $\Gamma_1' , \Gamma_2' \Rightarrow \Delta'$, there exists $C'$ which satisfies (1), (2) and (3) with parameters of the induction hypothesis, $\Gamma_1'$, $\Gamma_2'$ and $\Delta'$, in place of $\Gamma_1$, $\Gamma_2$ and $\Delta$ respectively. Notice that in all cases, we use induction hypothesis on the premises of $\D$, but in each case we instantiate the induction hypothesis with appropriate parameters we choose.

Here we will construct $C$ in the cases for the axioms, $LW$, $L \wedge ^n$, $R \wedge$, $L \rightarrow ^n$ and $R \rightarrow$ in details. The construction in the other cases is similar and not much hard to grasp.


\begin{enumerate}
	\item[($Id$)]
	\begin{enumerate}
		\item If $\Gamma_2 = p$, then take $C = \top$. So we have $\Gamma_1 \Rightarrow \top$ by $R \top$ and $p , \top \Rightarrow p$ by $Id$. The third condition is trivial.
		
		\item If $\Gamma_1 = p$ then take $C = p$.
	\end{enumerate}
	\item[($L \top$)] Take $C = \top$.
	
	\item[($R \bot$)] Take $C = \bot$.
	
	\item[($LW$)] Suppose $\D$ proves
	\begin{prooftree}
    \AXC{$\Gamma_1, \Gamma_2 \Rightarrow \Delta$}
    \RightLabel{$LW$}
    \UIC{$\Gamma_1, \Gamma_2, \Sigma_1, \Sigma_2 \Rightarrow \Delta$}
  \end{prooftree}
  and we want to construct $C$ such that $\Gamma_1, \Sigma_1 \Rightarrow C$ and $\Gamma_2, \Sigma_2, C \Rightarrow \Delta$ are provable and $V(C) \subseteq V(\Gamma_1, \Sigma_1) \cap V(\Gamma_2, \Sigma_2,\Delta)$. From induction hypothesis we have a $C$ such that $\Gamma_1 \Rightarrow C$ and $\Gamma_2 , C \Rightarrow \Delta$ are provable, which turn to the desired sequents using $LW$ again. From the induction hypothesis we also have $V(C) \subseteq V(\Gamma_1) \cap V(\Gamma_2, \Delta)$. One can observe $V(C) \subseteq V(\Gamma_1, \Sigma_1) \cap V(\Gamma_2, \Sigma_2, \Delta)$, since addind $\Sigma_1$ and $\Sigma_2$ only grows the sets of atoms on the sides of the intersection.

	\item[($L \wedge ^n$)] Suppose $\D$ proves
  \begin{prooftree}
    \AXC{$\Gamma_1, \Gamma_2, \nabla^n A, \nabla^n B \Rightarrow \Delta$}
    \RightLabel{$L \wedge ^n$}
    \UIC{$\Gamma_1 , \Gamma_2 , \nabla^n (A \wedge B) \Rightarrow \Delta$}
  \end{prooftree}
  There are two subcases, for whether (a) $\nabla^n (A \wedge B)$ should go with $\Gamma_1$ and appear in the sequent for the condition (1), or (b) $\nabla^n (A \wedge B)$  should go with $\Gamma_2$ and appear in the sequent for the condition (2).
	\begin{enumerate}
		\item From the induction hypothesis, take $C$ such that $\Gamma_1 \Rightarrow C$ and $\Gamma_2, \nabla^n A, \nabla^n B \Rightarrow \Delta$. So (1) is already satisfied, and (2) is also satisfied if we use a $L \wedge ^n$ on the latter sequent. It is easy to chcek that the condition (3) is also satisfied by induction hypothesis.

		\item From the induction hypothesis, take $C$ such that $\Gamma_1 , \nabla^n A, \nabla^n B \Rightarrow C$ and $\Gamma_2 \Rightarrow \Delta$ are provable. Apply $L \wedge ^n$ on the former sequent.
	\end{enumerate}

	\item[($L \rightarrow ^n$)] Suppose $\D$ proves
  \begin{prooftree}
    \AXC{$\Gamma_1, \Gamma_2, \nabla^{n+1} (A \rightarrow B) \Rightarrow \nabla^n A$}
    \AXC{$\Gamma_1, \Gamma_2, \nabla^{n+1} (A \rightarrow B), \nabla^n B \Rightarrow \Delta$}
    \RightLabel{$L \rightarrow ^n$}
    \BIC{$\Gamma_1 , \Gamma_2 , \nabla^{n+1} (A \rightarrow B) \Rightarrow \Delta$}
  \end{prooftree}
  Again, there are two further subcases for whether $\nabla^{n+1} (A \rightarrow B)$ should appear in the sequent for the condition (1) or (2).
	\begin{enumerate}
		\item We instantiate the induction hypothesis two times, with different parameters. First, take $C_1$ from the induction hypothesis such that $\Gamma_1 \Rightarrow C_1$ and $\Gamma_2, \nabla^{n+1} (A \rightarrow B), C_1 \Rightarrow \nabla^n A$ are provable. Second, take $C_2$ to be from another instance of the induction hypothesis where $\Gamma_1 \Rightarrow C_2$ and $\Gamma_2, \nabla^{n+1} (A \rightarrow B), \nabla^n B, C_2 \Rightarrow \Delta$ are provable. Construct $C$ as $C_1 \wedge C_2$.	We have $\Gamma_1 \Rightarrow C_1 \wedge C_2$ using $R \wedge$. By $LW$ and $L \wedge ^n$ we can also prove $\Gamma_2, \nabla^{n+1} (A \rightarrow B), C_1 \wedge C_2 \Rightarrow \nabla^n A$ and $\Gamma_2 , \nabla^n B, \nabla^{n+1} (A \rightarrow B), C_1 \wedge C_2 \Rightarrow \Delta$, to which we apply $L \rightarrow ^n$ to get to $\Gamma_2 , \nabla^{n+1} (A \rightarrow B) , C_1 \wedge C_2 \Rightarrow \Delta$.
		The condition (3) is easy to check.

		\item Again, we use two instances of the induction hypothesis with different parameters. From the induction hypothesis, take $C_1$ such that $\Gamma_2 \Rightarrow C_1$ and $\Gamma_1, \nabla^{n+1} (A \rightarrow B), C_1 \Rightarrow \nabla^n A$ are provable, and $C_2$ such that $\Gamma_1, \nabla^n B \Rightarrow C_2$ and $\Gamma_2, C_2 \Rightarrow \Delta$. Take $C = C_1 \supset C_2$. With a $LW$ we have $\Gamma_1, \nabla^n B, \nabla^{n+1} (A \rightarrow B), C_1 \Rightarrow C_2$. By $L \rightarrow ^n$ and $R \supset$ we get $\Gamma_1, \nabla^{n+1} (A \rightarrow B) \Rightarrow C_1 \supset C_2$.
		We also have $\Gamma_2, C_1 \supset C_2 \Rightarrow \Delta$ using $LW$ and $L \supset$. Again, the condition (3) holds as a direct result from the induction hypothesis.
	\end{enumerate}

	\item[($R \rightarrow$)] Suppose $\D$ proves
	\begin{prooftree}
    \AXC{$\nabla \Gamma_1, \nabla \Gamma_2, A \Rightarrow B$}
    \RightLabel{$R \rightarrow$}
    \UIC{$\Gamma_1, \Gamma_2 \Rightarrow A \rightarrow B$}
  \end{prooftree}
  Let $C_1$ be the interpolant from the induction hypothesis such that $\nabla \Gamma_1 \Rightarrow C_1$ and $\nabla \Gamma_2, A, C_1 \Rightarrow B$ are provable. Take $C = \top \rightarrow C_1$. By $LW$ and $R \rightarrow$ we have $\Gamma_1 \Rightarrow \top \rightarrow C_1$. On the other hand, we have $\nabla \Gamma_2, A \Rightarrow \top$ by $LW$ and $R \top$. Using $L \rightarrow ^n$ we get $\nabla \Gamma_2, A, \nabla (\top \rightarrow C_1) \Rightarrow B$. By $R \rightarrow$ we have $\Gamma_2, \top \rightarrow C_1 \Rightarrow A \rightarrow B$. Checking the condition (3) is straightforward, like the previous cases.
\end{enumerate}

Other cases should be handled similarly.
\end{proof}


We can not prove the same result for the fragment $\LDL_*$, because as we observed, the constructed interpolant uses $\supset$ in a case that we would eventually face, even if we are limited to the fragment without $\supset$. However, we can prove a weaker form of the interpolation theorem, called \emph{deductive interpolation}, in which $\vdash$ plays the role that $\Rightarrow$ has in interpolation.

\begin{nota}
  For the rest if this section, $\Gamma' \Rightarrow \Delta' \vdash \Gamma \Rightarrow \Delta$ means that there is a proof-tree in $\LDL_*$ for $\Gamma \Rightarrow \Delta$ with assumption $\Gamma' \Rightarrow \Delta'$.
\end{nota}

As we have seen, induction on the height of proof-trees is our main proof-theoretic tool here in this paper. So we must first make it clear what is the relation between $\vdash$ and $\Rightarrow$. Recall that $\Box A$ is a shorthand for $\top \rightarrow A$. We first define a \emph{variant} of a formula to be the formula with an arbitrary number of $\nabla$ and $\Box$.

\begin{dfn}
  The set of \emph{variants} of a formula $A$, is defined inductively as follows:
  \begin{enumerate}
    \item $A$ is a variant of $A$, and
    \item if $B$ is a variant of $A$, then $\nabla B$ and $\Box B$ are also variants of $A$.
  \end{enumerate}
  We say that a set of formulas $\Gamma$ is a variant of a set of formulas $\Sigma$, if any formula in $\Gamma$ is a variant of some formula in $\Sigma$.
\end{dfn}

To put it simply, variants of a formula $A$ are all possible combinations of $\nabla$ and $\Box$ applied on $A$.

In order to facilitate working with variants, we can define variations for the rules $R \rightarrow$ and $N$, which use $\Box$ instead of $\nabla$. The following lemma shows the admissibility of these variations.

\begin{lem}\label{lem:box-rules} For any $\Gamma$, $\Delta$, $A$, $B$ and $C$:
  \begin{enumerate}
    \item If $\LDL_* \vdash \Gamma, A \Rightarrow \Delta$ then $\LDL_* \vdash \Gamma, \nabla \Box A \Rightarrow \Delta$.
    \item If $\LDL_* \vdash \Gamma \Rightarrow \Delta$ then $\LDL_* \vdash \Box \Gamma \Rightarrow \Box \Delta$.
    \item If $\LDL_* \vdash \nabla \Gamma, \Sigma, A \Rightarrow B$ then $\LDL_* \vdash \Gamma, \Box \Sigma \Rightarrow A \rightarrow B$.
  \end{enumerate}
\end{lem}
Before proving the Lemma, we set a convention to use these admissible rules in our proof-trees.
\begin{nota}
  We will use the results from parts \1, \2 and \3 of Lemma \ref{lem:box-rules} as admissible rules $L \nabla \Box$, $\Box$ and $R \rightarrow'$, respectively.

  \begin{multicols}{3}
    \begin{prooftree}
      \AXC{$\Gamma, A \Rightarrow \Delta$}
      \RightLabel{$L \nabla \Box$}
      \UIC{$\Gamma, \nabla \Box A \Rightarrow \Delta$}
    \end{prooftree}
  \columnbreak
    \begin{prooftree}
      \AXC{$\Gamma \Rightarrow \Delta$}
      \RightLabel{$\Box$}
      \UIC{$\Box \Gamma \Rightarrow \Box \Delta$}
    \end{prooftree}
  \columnbreak
    \begin{prooftree}
      \AXC{$\nabla \Gamma, \Sigma, A \Rightarrow B$}
      \RightLabel{$R \rightarrow'$}
      \UIC{$\Gamma, \Box \Sigma \Rightarrow A \rightarrow B$}
    \end{prooftree}
  \end{multicols}
\end{nota}
Now we prove Lemma \ref{lem:box-rules}
\begin{proof} \quad\\
  \1
  \begin{prooftree}
    \AXC{$\Rightarrow \top$}
    \RightLabel{$LW$}
    \UIC{$\Gamma, \nabla \Box A \Rightarrow \top$}
    \AXC{$\Gamma, A \Rightarrow \Delta$}
    \RightLabel{$LW$}
    \UIC{$\Gamma, \nabla \Box A, A \Rightarrow \Delta$}
    \RightLabel{$L \rightarrow ^n$}
    \BIC{$\Gamma, \nabla \Box A \Rightarrow \Delta$}
  \end{prooftree}
  \begin{multicols}{2}
      \2
      \begin{prooftree}
        \AXC{$\Gamma \Rightarrow \Delta$}
        \RightLabel{$L \nabla \Box^{(*)}$} \doubleLine
        \UIC{$\nabla \Box \Gamma \Rightarrow \Delta$}
        \RightLabel{$LW$}
        \UIC{$\nabla \Box \Gamma, \top \Rightarrow \Delta$}
        \RightLabel{$R \rightarrow$}
        \UIC{$\Box \Gamma \Rightarrow \Box \Delta$}
      \end{prooftree}
    \columnbreak
      \3
      \begin{prooftree}
        \AXC{$\nabla \Gamma, \Sigma, A \Rightarrow B$}
        \RightLabel{$L \nabla \Box^{(*)}$} \doubleLine
        \UIC{$\nabla \Gamma, \nabla \Box \Sigma, A \Rightarrow B$}
        \RightLabel{$R \rightarrow$}
        \UIC{$\Gamma, \Box \Sigma \Rightarrow A \rightarrow B$}
      \end{prooftree}
  \end{multicols}{3}
\end{proof}

\begin{rem}\label{rem:var-val}
  Notice that any variation of a valid formula is also a valid formula. More precisely, we can prove any combination of $\nabla$ and $\Box$ applied on $A$, using the rules $N$ and $\Box$ applied to $\Rightarrow A$.
\end{rem}

The next theorem shows that every sequent provable using a formula as assumption, is also provable using some variant of that formula in the antecedent.

\begin{lem} \label{lem:vdash}
  For any set of formulas $A$, $\Gamma$ and $\Delta$, the following are equivalent:
  \begin{enumerate}
    \item $\Gamma, \Sigma_A \Rightarrow \Delta$, for a set of variants of $A$ like $\Sigma_A$.
    \item $\Rightarrow A \vdash \Gamma \Rightarrow \Delta$.
  \end{enumerate}
\end{lem}
\begin{proof}
  From \1 to \2. Suppose we have $\Gamma, \Sigma_A \Rightarrow \Delta$ and $\Rightarrow A$, for some sets of formulas $\Gamma$ and $\Delta$, some formula $A$ and a set of its variants $\Sigma_A$. As mentioned in Remark \ref{rem:var-val}, for any member of $\Sigma_A$ such as $A'$, we can prove $\Rightarrow A'$ using appropriate number of $N$ and $\Box$. Then we can use $cut$ with $\Gamma, \Sigma_A \Rightarrow \Delta$ to prove $\Gamma \Rightarrow \Delta$.

  From \2 to \1. Proceed by induction on the proof-tree $\D$ for $\Rightarrow A \vdash \Gamma \Rightarrow \Delta$ and in each case for its last rule, construct a proof-tree for $\Gamma, \Sigma_A \Rightarrow \Delta$, without the premise $\Rightarrow A$. First, suppose $\Gamma \Rightarrow \Delta$ is $\Rightarrow A$ itself. Take $A' = A$ and we have $A \Rightarrow A$ by $Id$. For the other cases for the last rule of $\D$, just take the proof-tree that the induction hypothesis on the premises, apply an appropriate instance of $LW$ if needed, and then apply the last rule of $\D$ again. We explaing some cases here, the rest are similar. Suppose $\Gamma \Rightarrow \Delta$ is proved by any of the following rules:

  \begin{enumerate}
    \item[$(L \rightarrow ^n)$] Suppose $\D$ ends with $\Gamma, \nabla^{n+1} (B \rightarrow C) \Rightarrow \Delta$, with two premises $\Gamma, \nabla^{n+1} (B \rightarrow C) \Rightarrow \nabla^n B$ and $\Gamma, \nabla^{n+1} (B \rightarrow C), \nabla^n C \Rightarrow \Delta$. From induction hypothesis, there are two sets $\Sigma_A'$ and $\Sigma_A''$ of variants of $A$, and two proof-trees (without the assumption $\Rightarrow A$) for $\Gamma, \Sigma_A', \nabla^{n+1} (B \rightarrow C) \Rightarrow \nabla^n B$ and $\Gamma, \Sigma_A'', \nabla^{n+1} (B \rightarrow C), \nabla^n C \Rightarrow \Delta$. Using $LW$, we can add $\Sigma_A''$ and $\Sigma_A'$ to the antecedent of the first and the second sequent, respectively. Using $L \rightarrow ^n$ again results in the desired sequent, with $\Sigma_A = \Sigma_A', \Sigma_A''$.

    \item[$(R \rightarrow)$] Suppose $\D$ ends with $\Gamma \Rightarrow B \rightarrow C$, and has a premise $\nabla \Gamma, B \Rightarrow C$. From induction hypothesis, we have a proof-tree (without the assumption $\Rightarrow A$) for $\nabla \Gamma, \Sigma_A', B \Rightarrow C$, for a set of variants of $A$ like $\Sigma_A'$. Apply $R \rightarrow'$ to get $\Gamma, \Box \Sigma_A' \Rightarrow B \rightarrow C$, which proves the claim, with $\Sigma_A = \Box \Sigma_A'$.

    \item[$(N)$] Suppose $\D$ ends with $\nabla \Gamma \Rightarrow \nabla \Delta$, and has a premise $\Gamma \Rightarrow \Delta$. From induction hypothesis, we have a proof-tree (without the assumption $\Rightarrow A$) for $\Gamma, \Sigma_A' \Rightarrow \Delta$, for a set of variants of $A$ like $\Sigma_A'$. Apply $N$ again, to get $\nabla \Gamma, \nabla \Sigma_A' \Rightarrow \nabla \Delta$, which proves the claim, with $\Sigma_A = \nabla \Sigma_A'$.
  \end{enumerate}
\end{proof}