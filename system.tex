The systems $\LDL$ and $\LDL_*$ are defined over the language $\L$ and $\L_*$, respectively. The system $\LDL$ is defined by the table of rules below, while the system $\LDL_*$ is defined by the same rules, except $L \supset$ and $R \supset$.
\begin{multicols}{3}
  \begin{prooftree}
    \AXC{}
    \RightLabel{$Id ^p$}
    \UIC{$p \Rightarrow p$}
  \end{prooftree}
  \columnbreak
  \begin{prooftree}
    \AXC{}
    \RightLabel{$L \bot$}
    \UIC{$\bot \Rightarrow$}
  \end{prooftree}
  \columnbreak
  \begin{prooftree}
    \AXC{}
    \RightLabel{$R \top$}
    \UIC{$\Rightarrow \top$}
  \end{prooftree}
\end{multicols}

\begin{multicols}{2}
  \begin{prooftree}
    \AXC{$ \Gamma \Rightarrow \Delta$}
    \RightLabel{$LW$}
    \UIC{$ \Gamma, \uwave{\Sigma} \Rightarrow \Delta$}
  \end{prooftree}
  \columnbreak
  \begin{prooftree}
    \AXC{$ \Gamma \Rightarrow$}
    \RightLabel{$Rw$}
     \UIC{$\Gamma \Rightarrow \uwave{A}$}		
  \end{prooftree}
 \end{multicols}

\begin{multicols}{2}
  \begin{prooftree}
    \AXC{$\Gamma, \nabla^n A, \nabla^n B \Rightarrow \Delta$}
    \RightLabel{$L \wedge ^n$}
    \UIC{$\Gamma, \uwave{\nabla^n (A \wedge B)} \Rightarrow \Delta$}		
  \end{prooftree}
  \columnbreak
  \begin{prooftree}
    \AXC{$\Gamma \Rightarrow A$}
    \AXC{$\Gamma \Rightarrow B$}
    \RightLabel{$R \wedge$}
    \BIC{$ \Gamma \Rightarrow \uwave{A \wedge B}$}		
  \end{prooftree}
\end{multicols}

\begin{prooftree}
  \AXC{$ \Gamma, \nabla^n A \Rightarrow \Delta$}
  \AXC{$\Gamma, \nabla^n B \Rightarrow \Delta$}
  \RightLabel{$L \vee ^n$}
  \BIC{$ \Gamma, \uwave{\nabla^n (A \vee B)} \Rightarrow \Delta$}		
\end{prooftree}

\begin{multicols}{2}
  \columnbreak
  \begin{prooftree}
    \AXC{$\Gamma \Rightarrow A$}
    \RightLabel{$R \vee_1$}
    \UIC{$\Gamma \Rightarrow \uwave{A \vee B}$}		
  \end{prooftree}
  \columnbreak
  \begin{prooftree}
    \AXC{$\Gamma \Rightarrow B$}
    \RightLabel{$R \vee_2$}
    \UIC{$\Gamma \Rightarrow \uwave{A \vee B}$}		
  \end{prooftree}
\end{multicols}


\begin{prooftree}
  \AXC{$\Gamma, \nabla^{n+1} (A \rightarrow B) \Rightarrow \nabla^n A$}
  \AXC{$\Gamma, \nabla^{n+1} (A \rightarrow B), \nabla^n B \Rightarrow \Delta$}
  \RightLabel{$L \rightarrow ^n$}
  \BIC{$\Gamma, \uwave{\nabla^{n+1} (A \rightarrow B)} \Rightarrow \Delta$}		
\end{prooftree}

\begin{prooftree}
  \AXC{$\nabla \Gamma, A \Rightarrow B$}
  \RightLabel{$R \rightarrow$}
  \UIC{$\Gamma \Rightarrow \uwave{A \rightarrow B}$}		
\end{prooftree}

\begin{prooftree}
  \AXC{$\Gamma, \nabla^n (A \supset B) \Rightarrow \nabla^n A$}
  \AXC{$\Gamma, \nabla^n B \Rightarrow \Delta$}
  \RightLabel{$L \supset ^n$}
  \BIC{$\Gamma, \uwave{\nabla^n (A \supset B)} \Rightarrow \Delta$}		
\end{prooftree}

\begin{prooftree}
  \AXC{$\Gamma, A \Rightarrow B$}
  \RightLabel{$R \supset$}
  \UIC{$\Gamma \Rightarrow \uwave{A \supset B}$}		
\end{prooftree}

\begin{prooftree}
  \AXC{$\Gamma \Rightarrow \Delta$}
  \RightLabel{$N$}
  \UIC{$\nabla \Gamma \Rightarrow \nabla \Delta$}
\end{prooftree}

We refer to the rules $Id ^p$, $L \bot$ and $R \top$ as the \emph{axioms}, $L \wedge ^n$, $R \wedge$, $L \vee ^n$, $R \vee_1$, $R \vee_2$, $L \rightarrow ^n$, $R \rightarrow$, $L \supset ^n$ and $R \supset$ as the \emph{logical rules}, and $LW$ and $RW$ as the \emph{structural rules}.

\begin{rem}\label{rem:ldl-gstn}
  Notice the differences between $\GSTN$ and $\LDL$. The latter lacks the rules $cut$ and $Lc$, its $Id ^p$ rule is just $Id$, limited to the atomic formulas, and its $LW$ can introduce more than one formula to its conclusion, which is admissible in $\GSTN$ by multiple instances of $Lw$. It also differs in $L \wedge ^n$, $L \vee ^n$, $L \rightarrow ^n$ and $L \supset ^n$, which also have arbitrary number of $\nabla$s on their principal formula, which is distributed over its subformulas in their premises.  
\end{rem}

\begin{rem}
  The system $\LDL$ is \emph{analytic}, in the sense that for all of its rules, all formulas in the premises are subformulas of its conclusion, up to $\nabla$. More percisely, all formulas in the premises are of the form $\nabla^n A$ (for some $n \geq 0$) where $A$ is a subformula of the conclusion.
\end{rem}

A direct consequence of the analyticity of $\LDL$ is that we don't need the rules $L \supset$ and $R \supset$ for proving sequents in the language $\L_*$.

\begin{thm}\label{thm:ldl-cons-ext}
  For any sequent $\Gamma \Rightarrow \Delta$ in the language $\L_*$, if $\LDL \vdash \Gamma \Rightarrow \Delta$ then $\LDL_* \vdash \Gamma \Rightarrow \Delta$.
\end{thm}
\begin{proof}
  Easy, by induction on the proof-tree for in $\LDL$. Notice that the cases for $L \supset$ and $R \supset$ are refuted by the assumption.
\end{proof}

In the rest of this section we are going to show that $\LDL$ and $\GSTN$ are equivalent, i.e. prove exactly the same set of sequents. Hence, proving that the logic $\ldl$ has an analytic system. To reach this aim, we have to show the admissibility of $Id$, $Lc$ and $cut$ from $\GSTN$ in $\LDL$, and the admissibility of $L \wedge ^n$, $L \vee ^n$, $L \rightarrow ^n$ and $L \supset ^n$ in $\GSTN$. We also need some lemmas which facilitate our admissibility proofs.

\begin{lem}\label{lem:l-nabla-dist} For each $n \geq 0$
	\begin{enumerate}
		\item $\GSTN \vdash \nabla^n (A \vee B) \Rightarrow \nabla^n A \vee \nabla^n B$.
	
		\item $\GSTN \vdash \nabla^n (A \wedge B) \Rightarrow \nabla^n A \wedge \nabla^n B$. 
	
		\item $\GSTN \vdash \nabla^n (A \rightarrow B) \Rightarrow \nabla^n A \rightarrow \nabla^n B$.
	\end{enumerate}
\end{lem}
\begin{proof}

  For (1), first, observe that

  \begin{prooftree}
    \AXC{}
    \RightLabel{$R \top$}
    \UIC{$\Rightarrow \top$}
    \RightLabel{$Lw$}
    \UIC{$\nabla (\top \rightarrow C) \Rightarrow \top$}
  
    \AXC{}
    \RightLabel{$Id$}
    \UIC{$C \Rightarrow C$}
    \RightLabel{$Lw$}
    \UIC{$\nabla (\top \rightarrow C), C \Rightarrow C$}
  
    \RightLabel{$L \rightarrow$}
    \BIC{$\nabla (\top \rightarrow C) \Rightarrow C$}
  \end{prooftree}
  
  
  Now, let $\D$ be the proof-tree above with $C = \nabla A \vee \nabla B$.
  
  \begin{prooftree}
    \AXC{}
    \RightLabel{$Id$}
    \UIC{$\nabla A \Rightarrow \nabla A$}
    \RightLabel{$R\vee_1$}
    \UIC{$\nabla A \Rightarrow \nabla A \vee \nabla B$}
    \RightLabel{$Lw$}
    \UIC{$\nabla A , \top \Rightarrow \nabla A \vee \nabla B$}
    \RightLabel{$R \rightarrow$}
    \UIC{$A \Rightarrow \top \rightarrow (\nabla A \vee \nabla B)$}
  
    \AXC{}
    \RightLabel{$Id$}
    \UIC{$\nabla B \Rightarrow \nabla B$}
    \RightLabel{$R\vee_2$}
    \UIC{$\nabla B \Rightarrow \nabla A \vee \nabla B$}
    \RightLabel{$Lw$}
    \UIC{$\nabla B , \top \Rightarrow \nabla A \vee \nabla B$}
    \RightLabel{$R \rightarrow$}
    \UIC{$B \Rightarrow \top \rightarrow (\nabla A \vee \nabla B)$}
  
    \RightLabel{$L\vee$}
    \BIC{$A \vee B \Rightarrow \top \rightarrow (\nabla A \vee \nabla B)$}
    \RightLabel{$N$}
    \UIC{$\nabla (A \vee B) \Rightarrow \nabla (\top \rightarrow (\nabla A \vee \nabla B))$}
  
    \AXC{$\D$}
  
    \RightLabel{$cut$}
    \BIC{$\nabla (A \vee B) \Rightarrow \nabla A \vee \nabla B$}
  \end{prooftree}
  
  Call this proof-tree $\D_1$. Construct $\D_{n+1}$ inductively as follows.
  
  \begin{prooftree}
    \AXC{$\D_n$}
    \noLine
    \UIC{$\nabla^n (A \vee B) \Rightarrow \nabla^n A \vee \nabla^n B$}
    \RightLabel{$N$}
    \UIC{$\nabla^{n+1} (A \vee B) \Rightarrow \nabla (\nabla^n A \vee \nabla^n B)$}
  
    \AXC{$\D_1$}
    \noLine
    \UIC{$\nabla (\nabla^n A \vee \nabla^n B) \Rightarrow \nabla^{n+1} A \vee \nabla^{n+1} B$}
    
    \RightLabel{$cut$} \LeftLabel{$\D_{n+1}:$}
    \BIC{$\nabla^{n+1} (A \vee B) \Rightarrow \nabla^{n+1} A \vee \nabla^{n+1} B$}
  \end{prooftree}


  For (2) and (3) consider the following proof-trees:
  \begin{prooftree}
    \AXC{}
    \RightLabel{$Id$}
    \UIC{$A \Rightarrow A$}
    \RightLabel{$L \wedge_1$}
    \UIC{$A \wedge B \Rightarrow A$}
    \RightLabel{$N^{(*)}$} \doubleLine
    \UIC{$\nabla^n (A \wedge B) \Rightarrow \nabla^n A$}
  
    \AXC{}
    \RightLabel{$Id$}
    \UIC{$B \Rightarrow B$}
    \RightLabel{$L \wedge_2$}
    \UIC{$A \wedge B \Rightarrow B$}
    \RightLabel{$N^{(*)}$} \doubleLine	
    \UIC{$\nabla^n (A \wedge B) \Rightarrow \nabla^n B$}
    
    \RightLabel{$R \wedge$}
    \BIC{$\nabla^n (A \wedge B) \Rightarrow \nabla^n A \wedge \nabla^n B$}
  \end{prooftree}
\vspace*{1cm}
  \item \quad
  \begin{prooftree}
    \AXC{}
    \RightLabel{$Id$}
    \UIC{$A \Rightarrow A$}
    \RightLabel{$Lw$}
    \UIC{$A, \nabla (A \rightarrow B) \Rightarrow A$}
  
    \AXC{}
    \RightLabel{$Id$}
    \UIC{$B \Rightarrow B$}
    \RightLabel{$Lw$}
    \UIC{$A, \nabla (A \rightarrow B), B \Rightarrow B$}
  
    \RightLabel{$L \rightarrow$}
    \BIC{$\nabla (A \rightarrow B) , A \Rightarrow B$}
    \RightLabel{$N^{(*)}$} \doubleLine
    \UIC{$\nabla^{n+1} (A \rightarrow B) , \nabla^n A \Rightarrow \nabla^n B$}
    \RightLabel{$R \rightarrow$}
    \UIC{$\nabla^n (A \rightarrow B) \Rightarrow \nabla^n A \rightarrow \nabla^n B$}
  \end{prooftree}
  where $N^{(*)}$ denotes multiple uses of the rule $N$.
\end{proof}


\begin{thm}\label{thm:id-adm}
	$\LDL \vdash A \Rightarrow A$.
\end{thm}
\begin{proof}
	By induction on the structure of $A$. If $A$ is an atom, $\bot$ or $\top$, we would have $A \Rightarrow A$ using the axioms of $\LDL$, $Lw$ and $Rw$. If $A = B \circ C$, where $\circ \in \{\wedge, \vee, \supset\}$, the claim is clear from the induction hypothesis for $B$ and $C$. We will explain the proof in details for $A = B \rightarrow C$ and $A = \nabla B$.

If $A = B \rightarrow C$, then we have $B \rightarrow C \Rightarrow B \rightarrow C$ by the following proof-tree. The subtrees denoted by IH are obtained from the induction hypothesis.
\begin{prooftree}
  \AXC{IH} \noLine
  \UIC{$B \Rightarrow B$} \RightLabel{$Lw$}
  \UIC{$\nabla (B \rightarrow C), B \Rightarrow B$}
  \AXC{IH} \noLine
  \UIC{$C \Rightarrow C$} \RightLabel{$Lw$}
  \UIC{$\nabla (B \rightarrow C), C \Rightarrow C$}
  \RightLabel{$L \rightarrow ^n$}
  \BIC{$\nabla (B \rightarrow C), B \Rightarrow C$}
  \RightLabel{$R \rightarrow$}
  \UIC{$B \rightarrow C \Rightarrow B \rightarrow C$}
\end{prooftree}

If $A = \nabla B$, then we can prove $\nabla B \Rightarrow \nabla B$ by applying $N$ on $B \Rightarrow B$, which is proved using induction hypothesis.

\end{proof}

In order to prove the admissibility of $Lc$, we also need the following \emph{inversion} lemmas for some of the rules of $\LDL$.

\begin{lem}[Inversion]\label{lem:inv} \quad
	\begin{enumerate}
		\item If $\LDL \vdash_h \Gamma, \nabla^n (A \wedge B) \Rightarrow \Delta$ then $\LDL \vdash_h \Gamma, \nabla^n A, \nabla^n B \Rightarrow \Delta$.
		\item If $\LDL \vdash_h \Gamma, \nabla^n (A \vee B) \Rightarrow \Delta$ then $\LDL \vdash_h \Gamma, \nabla^n A \Rightarrow \Delta$ and $\LDL \vdash_h \Gamma, \nabla^n B \Rightarrow \Delta$.
  	\item If $\LDL \vdash_h \Gamma, \nabla^n (A \supset B) \Rightarrow \Delta$ then $\LDL \vdash_h \Gamma, \nabla^n B \Rightarrow \Delta$.
	\end{enumerate}
\end{lem}
\begin{proof}
  We only explain the proof for (3), the other two parts are similar. We use induction on $h$, the height of the proof-tree for $\Gamma, \nabla^n (A \supset B) \Rightarrow \Delta$. It is clear that $h$ can not be $0$, since $\Gamma, \nabla^n (A \supset B) \Rightarrow \Delta$ can not be proved by an axiom. If the proof-tree ends with $L \supset ^n$, and $\nabla^n (A \supset B)$ is the principal formula, then the claim would be obvious. In any other case, just commute the last rule with the sequent obtained from the induction hypothesis. Here we explain the case for $N$.

  Suppose the last rule is $N$. So we have
  \begin{prooftree}
    \AXC{$\Gamma, \nabla^n (A \supset B) \Rightarrow \Delta$} \RightLabel{$N$}
    \UIC{$\nabla \Gamma, \nabla^{n+1} (A \supset B) \Rightarrow \nabla \Delta$}
  \end{prooftree}
  at the end of a proof-tree of height $h > 0$. From the induction hypothesis for the premise, we have a proof for $\Gamma, \nabla^n B \Rightarrow \Delta$ of heigt $h - 1$. Using $N$, we have a proof-tree for $\nabla \Gamma, \nabla^{n+1} B \Rightarrow \nabla \Delta$ with height $h$.

  The cases for the other rules are similar.
\end{proof}

The next theorem shows that the rule $Lc$ from $\GSTN$ is admissible in $\LDL$.
\begin{thm}[Height-preserving admissibility of $Lc$]\label{thm:lc-adm} For any sequent $\Gamma$, $\Delta$ and $A$, and any $h \geq 0$, if $\LDL \vdash_h \Gamma, A, A \Rightarrow \Delta$ then $\LDL \vdash_h \Gamma, A \Rightarrow \Delta$.
\end{thm}
\begin{proof}
  By induction on $h$. For $h = 0$, the proof for $\Gamma, A, A \Rightarrow \Delta$ must be an axiom, which is impossible. For $h > 0$, we investigate different cases by the last rule of the proof of $\Gamma, A, A \Rightarrow \Delta$. The cases where none of occurrences of $A$ are principal are trivial, as it is enough to apply the same rule on the proofs the induction hypothesis provides.
  
  If one occurrence of $A$ is principal, we have the following cases based on the last rule of the proof:
	\begin{itemize}
		\item[$(L \wedge ^n)$] If the last rule is $L \wedge ^n$, then $A = \nabla^n (B \wedge C)$ it is in the form:

		\begin{prooftree}
			\AXC{$\D_0$} \noLine
			\UIC{$\Gamma, \nabla^n B, \nabla^n C, \nabla^n (B \wedge C) \Rightarrow \Delta$}
			\RightLabel{$L \wedge ^n$}
			\UIC{$\Gamma, \nabla^n (B \wedge C), \nabla^n (B \wedge C) \Rightarrow \Delta$}		
		\end{prooftree}

		Then, by Lemma \ref{lem:inv} we have a proof for $\Gamma, \nabla^n B, \nabla^n B, \nabla^n C, \nabla^n C \Rightarrow \Delta$ of height less than $h$. Using induction hypothesis twice, we would have a proof for $\Gamma, \nabla^n B, \nabla^n C \Rightarrow \Delta$ of height less than $h$. Finally, $L \wedge ^n$ will result in the desired proof with height at most $h$.
	
		\item[$(L \vee ^n)$] If the last rule is $L \vee ^n$, then $A = \nabla^n (B \vee C)$ and the last rule is in the form:
		\begin{prooftree}
			\AXC{$\D_0$} \noLine
			\UIC{$ \Gamma, \nabla^n B, \nabla^n (B \vee C) \Rightarrow \Delta$}
			\AXC{$\D_1$} \noLine
			\UIC{$\Gamma, \nabla^n C, \nabla^n (B \vee C) \Rightarrow \Delta$}
			\RightLabel{$L \vee ^n$}
			\BIC{$ \Gamma, \nabla^n (B \vee C), \nabla^n (B \vee C) \Rightarrow \Delta$}		
		\end{prooftree}
		By Lemma \ref{lem:inv} for $\D_0$ and $\D_1$, we have proofs for $\Gamma, \nabla^n B, \nabla^n B \Rightarrow \Delta$ and $\Gamma, \nabla^n C, \nabla^n C \Rightarrow \Delta$ of heights less than $h$. By induction hypothesis, we have proofs for $\Gamma, \nabla^n B \Rightarrow \Delta$ and $\Gamma, \nabla^n C \Rightarrow \Delta$ of heights less than $h$. By $L \vee ^n$ we have the desired proof with height at most $h$.
	
		\item[$(L \rightarrow ^n)$] If the last rule is $L \rightarrow ^n$, then $A = \nabla^{n+1} (B \rightarrow C)$ and the last rule is in the form:
		\begin{prooftree}
			\AXC{$\D_0$} \noLine
			\UIC{$\Gamma, \nabla^{n+1} (B \rightarrow C), \nabla^{n+1} (B \rightarrow C) \Rightarrow \nabla^n B$}
			\AXC{$\D_1$} \noLine
			\UIC{$\Gamma, \nabla^n C, (\nabla^{n+1} (B \rightarrow C)), \nabla^{n+1} (B \rightarrow C) \Rightarrow \Delta$}
			\RightLabel{$L \rightarrow ^n$}
			\BIC{$\Gamma, \nabla^{n+1} (B \rightarrow C), \nabla^{n+1} (B \rightarrow C) \Rightarrow \Delta$}
		\end{prooftree}
		By induction hypothesis, we have a proof for $\Gamma, \nabla^{n+1} (B \rightarrow C) \Rightarrow \nabla^n B$ and $\Gamma, \nabla^n C, \nabla^{n+1} (B \rightarrow C) \Rightarrow \Delta$ with height less than $h$. Finally, by $L \rightarrow ^n$ we have the desired proof with height at most $h$.

	
		\item[$(L \supset ^n)$] If the last rule is $L \supset ^n$, then $A = \nabla^n (B \supset C)$ and the last rule is in the form:
		\begin{prooftree}
			\AXC{$\D_0$} \noLine
			\UIC{$\Gamma, \nabla^n (B \supset C), \nabla^n (B \supset C) \Rightarrow \nabla^n B$}
			\AXC{$\D_1$} \noLine
			\UIC{$\Gamma, \nabla^n C, \nabla^n (B \supset C) \Rightarrow \Delta$}
			\RightLabel{$L \supset ^n$}
			\BIC{$\Gamma, \nabla^n (B \supset C), \nabla^n (B \supset C) \Rightarrow \Delta$}
		\end{prooftree}
		By Lemma \ref{lem:inv} for $\D_1$, we have a proof for $\Gamma, \nabla^n C, \nabla^n C \Rightarrow \Delta$ of height less than $h$. By induction hypothesis, we have proofs for $\Gamma, \nabla^n (B \supset C) \Rightarrow \nabla^n B$ and $\Gamma, \nabla^n C \Rightarrow \Delta$, both with heights less than $h$. Finally, by $L \supset ^n$ we have the desired proof, with height at most $h$.
	\end{itemize}
\end{proof}

It remains to show that $cut$ is admissible in $\LDL$.

\begin{thm}\label{thm:cut-adm}
  If $\LDL \vdash \Gamma \Rightarrow A$ and $\LDL \vdash \Sigma, A \Rightarrow \Delta$, then $\LDL \vdash \Gamma, \Sigma \Rightarrow \Delta$.
\end{thm}

We postpone the proof of Theorem \ref{thm:cut-adm} to the next section. Nevertheless, we will use its result to show that $\LDL$ and $\GSTN$ are equivalent, and therefore, share the same logic $\ldl$.

\begin{nota}
  Thanks to Theorems \ref{thm:id-adm}, \ref{thm:lc-adm} and \ref{thm:cut-adm}, from now on, we pretend that the rules $Id$, $Lc$ and $cut$ are available in $\LDL$ as well. Therefore, we use them in the proof-trees in $\LDL$ without any further explanation.
\end{nota}

Now, we are ready to prove the equivalence of two systems.

\begin{thm}\label{thm:ldl-eq-gstn}
  Any sequent $\Gamma \Rightarrow \Delta$ in the language $\L$ is provable in $\GSTN$ iff it is provable in $\LDL$.
\end{thm}
\begin{proof}
  By induction on the proof-tree for $\Gamma \Rightarrow \Delta$, the claim is easy to observe in both directions. For the left to right direction, as was observed in Remark \ref{rem:ldl-gstn}, it suffices to use Theorems \ref{thm:id-adm}, \ref{thm:lc-adm} and \ref{thm:cut-adm}. For the other direction, use Lemma \ref{lem:l-nabla-dist}. We will explain the case where the $\LDL$ proof-tree for $\Gamma \Rightarrow \Delta$ ends with $L \rightarrow ^n$. In this case, by inductin hypothesis, we have proof-trees for $\Gamma, \nabla^{n+1} (A \rightarrow B) \Rightarrow \nabla^n A$ and $\Gamma, \nabla^{n+1} (A \rightarrow B), \nabla^n B \Rightarrow \Delta$ in $\GSTN$. We also have $\nabla^n (A \rightarrow B) \Rightarrow \nabla^n A \rightarrow \nabla^n B$ from Lemma \ref{lem:l-nabla-dist} part \3.

  \begin{prooftree}
    \AXC{$\nabla^n (A \rightarrow B) \Rightarrow \nabla^n A \rightarrow \nabla^n B$}
    \RightLabel{$N$}
    \UIC{$\nabla^{n+1} (A \rightarrow B) \Rightarrow \nabla (\nabla^n A \rightarrow \nabla^n B)$}

    \AXC{$\Gamma, \nabla^{n+1} (A \rightarrow B) \Rightarrow \nabla^n A$}
    \AXC{$\Gamma, \nabla^{n+1} (A \rightarrow B), \nabla^n B \Rightarrow \Delta$}
    \RightLabel{$R \rightarrow$}
    \BIC{$\Gamma, \nabla^{n+1} (A \rightarrow B), \nabla (\nabla^n A \rightarrow \nabla^n B) \Rightarrow \Delta$}

    \RightLabel{$cut$}
    \BIC{$\Gamma, (\nabla^{n+1} (A \rightarrow B))^2 \Rightarrow \Delta$}
    \RightLabel{$Lc$}
    \UIC{$\Gamma, \nabla^{n+1} (A \rightarrow B) \Rightarrow \Delta$}
  \end{prooftree}

  Other cases are similar.
\end{proof}

\begin{cor}
  $\ldl$ is the logic of the system $\LDL$.
\end{cor}

Courtesy of these results and the fact that $\ldl$ is a conservative extension of $\ldl_*$, we have the same result for the systems without the rules $L \supset$ and $R \supset$.

\begin{thm}
  Any sequent $\Gamma \Rightarrow \Delta$ in the language $\L_*$ is provable in $\GSTN_*$ iff it is provable in $\LDL_*$.
\end{thm}
\begin{proof}
  Obviously, a proof-tree in $\GSTN_*$ or $\LDL_*$ is also a proof-tree in $\GSTN$ and $\LDL$, respectively. Use Theorem \ref{thm:ldl-eq-gstn} to translate proof-trees between $\GSTN$ and $\LDL$, then use Theorems \ref{thm:gstn-cons-ext} or \ref{thm:ldl-cons-ext} to translate proof-trees to $\LDL_*$ or $\GSTN_*$.
\end{proof}

So we can say the same about $\ldl_*$.

\begin{cor}
  $\ldl_*$ is the logic of the system $\LDL_*$.
\end{cor}