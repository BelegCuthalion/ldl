The systems $\LDL$ and $\LDL_*$ are defined over the language $\L$ and $\L_*$, respectively. The system $\LDL$ is defined by the table of rules below, while the system $\LDL_*$ is defined by the same rules, except $L \supset$ and $R \supset$.
\documentclass[10pt,a4paper]{amsart}
\usepackage{mathabx}
\usepackage[mathscr]{eucal}
\usepackage{bussproofs}
\EnableBpAbbreviations
\usepackage{amssymb}
\usepackage{enumitem}
\usepackage{multicol}
\usepackage[normalem]{ulem}

\theoremstyle{plain}
\newtheorem{thm}{Theorem}[section]

\renewcommand{\thethm}{\arabic{section}.\arabic{thm}}
\newtheorem{lem}[thm]{Lemma}

\newtheorem{cor}[thm]{Corollary}
\theoremstyle{definition}
\newtheorem{dfn}[thm]{Definition}
\newtheorem{exam}[thm]{Example}
\newtheorem{rem}[thm]{Remark}
\newtheorem{nota}[thm]{Notation}
\newtheorem{exer}[thm]{Exercise}

\def\d{\displaystyle}
\def\L{\mathcal{L}}
\def\ldl{\mathsf{LDL}}
\def\LDL{\mathbf{LDL}}
\def\iSTLN{\mathbf{iSTL(N)}}
\def\GSTN{\mathbf{GSTN}}

\def\D{\mathcal{D}}
\def\IH{\mathrm{IH}}
\def\1{\emph{(1)}}
\def\2{\emph{(2)}}
\def\3{\emph{(3)}}
\def\4{\emph{(4)}}

\begin{document}

\title{Interpolation for the Logic of Dynamic Locales}
\author{Amirhossein Akbar Tabatabai}


\author{Majid Alizadeh}

\author{Alireza Mahmoudian}

\date{\today}
 
\begin{abstract}
	A generalization of the notion of implication is introduced in \cite{amir}, as a binary operator which internalizes the reflexive and transitive properties of the provability relation on the set of all propositions. This generalization gives rise to a classification of some sub-intuitionistic and sub-structural logics with various behaviors of their respecting implication operators. The canonical instance of this generalization, which is \emph{the dynamic locale}, can be thought of as an algebraic model for \emph{dynamic topological systems}.

In this paper we will introduce a cut-free G3-style sequent calculus for the logic of dynamic locales and its extension with intuitionistic implication. Then we will study some properties of this system such as subformula property, disjuction property, admissibility of the Visser's rule and Craig's and deductive interpolation.
\end{abstract}

\maketitle

\keywords{Cut-elimination, Interpolation, Dynamics in general topological spaces}

\subjclass{03F05, 03C40, 37B02}

\section{Introduction}
As a logical connective, \emph{the implication} is supposed to capture the higher order notion of \emph{logical entailment}. So it must reflect properties of the entailment relation in the language level. But we may have a broad meaning of an entailment relation in mind, to include different logics. Also we may choose which set of properties we expect the implication to reflect, and which structures we want it to internalize. This gives rise to many different instances of an implication; Indeed, for example, the implication in classical, intuitionistic, substructural or basic logic behave differently.

For a more particular example, observe the usual behaviour that we expect from the implication operator in intuitionistic or classical logic: It internalizes properties such as reflexivity, transitivity, and all other properties of the entailment relation which are common in both intuitionistic and classical settings, like how it deals with the meaning of operators like conjunction or disjunction. For example, both classical and intuitionistic entailments enjoy a \emph{deduction theorem}, i.e. for some propositions $A$ and $B$, the metalogical relation of entailment between $A$ and $B$ is equivalent to the truth of the proposition $A \rightarrow B$.

On the other hand, since there are intrinsic differences between what we intend by intuitionistic and classical entailments, their respecting implication connective also behaves differently in many ways. For instance, one can observe that the validity of $\varphi = \neg \neg A \rightarrow A$ can be interpreted as the internalization of RAA: If the negation of $A$ entails absurdity, then $A$ is true. So, contrary to classical implication, the intuitionistic implication must reflect the fact that RAA is not a valid rule of inference in intuitionistic setting by refuting $\varphi$.

This leaves us with the question that what is the essence of internalizing the entailment relation, whihc is a preorder relation in its most general form. The first author investigates this question in \cite{amir}, by introducing \emph{abstract implication}, a binary operator general enough to include the definitions of all the implication connectives for the logics that was mentioned above. An abstract implication is thus defined as an operator which internalizes the reflexive and transitive nature of the entailment.

Although the definition of the abstract implication captures the preorder structure of the entailment relation between the propositions, yet it does not require the operator to participate in the well-known adjunction between implication and conjunction. So abstract implication include not only the well-behaved implication operators of intuitionistic or classical logic, but also operators for which the mentioned \emph{deduction theorem} does not hold necessarily.

A smaller class of abstract implications are also studied in \cite{amir} which do not break the adjunction between implication and conjunction. These implications coincide with the natural notion of implication in \emph{dynamic locales}, which are point-free version of topological spaces with a topological automorphism.

In light of these observations, \cite{amir} introduces sequent style systems for dynamic locales, along with different semantics for them. An extensive investigation of these systems is done from an algebraic and topological point of view, while proof-theoretic approaches fail in absence of a reasonable \emph{analyticity} criterion. The systems in \cite{amir} are not analytic in the sense that a proof-tree can have formulas that are not subformulas of the formulas in the consequence.

In this paper we will talk about the logic of dynamic locals, $\ldl$, and some of its extensions. In the next section, we will see some preliminaries and the definition of a sequent style system for the logic of dynamic locales, along with its algebraic semantics, as it is defined in \cite{amir}.
In the third section, we will introduce a sequent style, conraction-free and cut-free system for the logic of dynamic locales.
In the fourth section, the admissibility of cut in this system will be proved. Then, in section five, we will use this theorem to deduce some results about the logic of dynamic locales and some of its extensions, such as subformula property, disjuction property and admissibility of the generalized Visser's rule.
And in the last section, we will show that some extensions of $\ldl$ have the Craig and deductive interpolation property.

\section{Preliminaries}
In the following, a formula of a language is a free construction of the language, over a countable set of atomic variables. We define the language $\L_* = \{\rightarrow, \vee, \wedge, \nabla, \top, \bot\}$, and the extended language $\L = \L_* \cup \{\supset\}$. Constructors $\bot$, $\top$, $\nabla$, $\vee$, $\wedge$, $\rightarrow$ and $\supset$ are called operators, and have arities $0$, $0$, $1$, $2$, $2$, $2$ and $2$, respectively. We might also use the shorthand $\Box A$ for $\top \rightarrow A$.
$\Gamma$, $\Sigma$, $\Delta$ and $\Pi$ are finite multi-sets of formulas, $A$, $B$ and $C$ are formulas, $p$ is an atom and $n$, $l$, $r$ and $k$ are natural numbers. We denote by $V$ the set of all atoms, and by $V(A)$ and $V(\Gamma)$ the set of those that occur in the formula $A$ and in all formulas in $\Gamma$ respectively.

We will write $A$ for the singleton $\{A\}$ wherever it is inferable from the context.
Single-comma (``$,$'') is used as the multi-set union. We will also write $\Gamma^n$ for $n$ times union of a multi-set $\Gamma$ with itself. So we will write $\Gamma, A^2$ instead of $\Gamma, \{A, A\}$. Multiple applications of an operator such as $\nabla$ on a formula $A$ is denoted by $\nabla^n A$. Likewise, $\nabla^n \Gamma$ is the multi-set $\{ \nabla^n A \mid A \in \Gamma \}$.


A sequent $\Gamma \Rightarrow \Delta$ is a binary relation between $\Gamma$, a multi-set of formulas, called \emph{the antecedent} or \emph{the left-side}, and $\Delta$, a set of at most one formula, called \emph{the succedent} or \emph{the right-side}.

A rule $R$ is expressed as a relation between a set of $n$ sequents $\{ \Gamma_i \Rightarrow \Delta_i \mid 1 \leq i \leq n \}$ called \emph{premises} and a sequent $\Gamma \Rightarrow \Delta$ called \emph{conclusion}, and is written as follows:
\begin{prooftree}
  \AXC{$\Gamma_1 \Rightarrow \Delta_1$}
  \AXC{$\dots$}
  \AXC{$\Gamma_n \Rightarrow \Delta_n$}
  \RightLabel{$R$}
  \TIC{$\Gamma \Rightarrow \Delta$}
\end{prooftree}

A rule with $n = 0$ is called an \emph{axiom}. We will designate a specific formula in the conclusion of some rules, called \emph{the principal formula} of that rule, which will be indicated by writing it \uwave{underlined}.

A proof-tree in a system is defined in the usual way.
We will name proof-trees by $\D$, $\D'$ and so on, unless otherwise stated. We will name subtrees of $\D$ by $\D_0$, $\D_1$ and so on.
When a proof-tree, with $\Gamma \Rightarrow \Delta$ as its root, is constructed using a system $\mathbf{G}$, we say that $\mathbf{G}$ proves $\Gamma \Rightarrow \Delta$, and write it as $\mathbf{G} \vdash \Gamma \Rightarrow \Delta$.

We define the height of a proof-tree to be the length of its longest branch.
We will write $\mathbf{G} \vdash_h \Gamma \Rightarrow \Delta$ to indicate the existence of a proof-tree of height $h$ for $\Gamma \Rightarrow \Delta$ in system $\mathbf{G}$. We will also write $h(\D)$ for the height of a proof-tree $\D$.

The set of all formulas $A$ where $\mathbf{G} \vdash \Rightarrow A$ is called \emph{the logic} $\mathbf{G}$. We will use the same notation $\mathsf{L} \vdash A$ to denote the provability in a logic $\mathsf{L}$. We write the name for a logic in $\mathsf{sans~serif}$ to distinguish it from a system.

We will define an abstract implication as a binary operator which ``internalizes'' the preorder structure of a meet-semilatice. This will generalize a wide range of known implication operators, most notably, intuitionistic and classical implications.
\begin{dfn}
  Let $\mathcal{A} = (A, \le, \wedge, 1)$ be a bounded meet-semilatice, that is a strucure ordered with $\le$ and with $\wedge$ as the binary infimum (also called \emph{meet}) operator and $1$ as the maximum element. An abstract implication on $\mathcal{A}$ is a monotone operator $\rightarrow : \mathcal{A}^{op} \times \mathcal{A} \rightarrow \mathcal{A}$ which satisfies the following:
  \begin{enumerate}
    \item $a \rightarrow a = 1$
    \item $(a \rightarrow b) \wedge (b \rightarrow c) \le (a \rightarrow c)$
  \end{enumerate}
  Then the structure $(\mathcal{A}, \le, \rightarrow, \wedge, 1)$ is called a \emph{strong algebra}.
  \end{dfn}
  Observe that both implications in the intuitionistic and classical settings satisfy this definition, capturing the preorder structure of entailment over proppsitions in their respective semantics, namely, Heyting and Boolean algebras. We can also find very weak instances of this definition. For example, define $a \rightarrow b = 1$, for all $a$ and $b$. These examples show that some desired properties of an implication do not necessarily hold for abstract implications, such as the adjunction $(\_) \wedge x \dashv x \rightarrow (\_)$, which means that these weak implications will not have the usual pair of introduction-elimination rules. This shows that, from an algebraic point of view, abstract implication is not generally well-behaved.
  
  The interesting instance of an abstract implication, which is also logically well-behaved, is that of a \emph{$\nabla$-algebra}, where the implication is a part of an adjunction.

  \begin{dfn}\label{dfn:n-alg} Let $(A, \le, \vee, \wedge, 0, 1)$ be a bounded lattice, which is an ordered structure with $\vee$ and $\wedge$ as the binary supremum (also called \emph{join}) and the binary infimum operators, and $0$ and $1$ are the minimum and maximum elements. A \emph{$\nabla$-algebra} is a structure $(A, \le, \rightarrow, \vee, \wedge, \nabla, 0, 1)$ where $\rightarrow$ and $\nabla$ are binary and unary operations respectively, satisfying the adjunction below, for all $a$, $b$ and $c \in A$.
  \[ \nabla c \wedge a \le b \iff c \le a \rightarrow b \]
  It follows from this definition that $\nabla$ is monotone and join-preserving. A $\nabla$-algebra is called \emph{normal} if $\nabla$ also meet-preserving.
  \end{dfn}

  It is also not hard to see that the operator $\rightarrow$ in any $\nabla$-algebra is an implication. However, as shown in \cite{amir}, any abstract implication can be represented by the well-behaved implications of $\nabla$-algebras:

  \begin{thm}
  For any strong algebra $(A, \le_A, \rightarrow_A, \wedge_A, 1_A)$, there exists a $\nabla$-algebra $(B, \le_B, \rightarrow_B, \vee_B, \wedge_B, 0_B, 1_B)$, a monotone function $F : B \to B$, and a bounded meet-semilatice embedding $i : A \to B$ (i.e., an injection which preserves the bounded meet semi-lattice structure) such that $i(a \rightarrow_A b) = F(i(a)) \rightarrow_B F(i(b))$, for all $a$ and $b \in A$.
  \end{thm}

  As a result, $\nabla$-algerbras could be seen as the algebraically well-behaved counterpart for strong algebras, which motivates studying the former. Also, $\nabla$-algebras a natural source of the abstract implications.
  \vspace*{2mm}
  
  Another motivation for studying $\nabla$-algebras are from a \emph{dynamic topological system}, which is a topological space augumented with a unary continuous function over it. The algebraic version of a topological space is a \emph{locale}, which is a complete lattice (that is a lattice with all suprema and infima for its arbitrary subsets, hence, necessarily bounded) whose binary meet operator distributes over arbitrary joins. A \emph{dynamic locale} is locale augumented with a localic map, which is a morphism that preserves binary meets and arbitrary joins.
  
  The canonical example of locales and the localic maps are the lattice of the open subsets of a topological space and the inverse image of the continuous functions between spaces, respectively. In this sense, locales and the localic maps provide a point-free formalization for the topological discourse, focusing only on open subsets as the main ingredient of topology.

  \begin{dfn}
    Let $\mathcal{X} = (X, \le, \vee, \wedge, 0, 1)$ be a complete lattice. We call $\mathcal{X}$ a \emph{locale} if for all $x \in X$ and $Y \subseteq X$, we have $x \wedge \bigvee_{y \in Y} y = \bigvee_{y \in Y} (x \wedge y)$.
    A \emph{dynamic locale} is a structure $(X, \le, \wedge, \vee, \nabla, 0, 1)$ where $(X, \le, \vee, \wedge, 0, 1)$ is a locale and $\nabla : X \rightarrow \mathcal X$ is monotone and preserves binary meet and arbitrary joins.
  \end{dfn}

  It is also shown that on a dynamic locale, an implication is induced which satisfies the adjunction in Definition \ref{dfn:n-alg}. Hence, one can observe that a dynamic locale is essentially a $\nabla$-algebra where we replaced the underlying bounded lattice structure with a locale, and whose $\nabla$ operator is also meet-preserving. The next theorem which states this fact is proved in \cite{amir}.
  
  \begin{thm}
    Let $\mathcal{X} = (X, \le, \wedge, \vee, \nabla, 0, 1)$ be a dynamic locale. There is an implication $\rightarrow$ over $X$ where $(X, \le, \rightarrow, \wedge, \vee, \nabla, 0, 1)$ is a $\nabla$-algebra.
  \end{thm}

  Notice that there is a Heyting structure in a locale, and likewise in a dynamic locale. We can define the usual \emph{heyting implication}, as $a \supset b = \bigvee \{c \mid c \wedge a \le b\}$. It is convenient to add implications from both $\nabla$-algebra structure and Heyting structure to the signature of a dynamic locale, where it is needed. So we might write a locale as a tuple $(X, \le, \rightarrow, \supset, \wedge, \vee, \nabla, 0, 1)$.

  Now, we need to define what is means for a sequent to be true in a dynamic locale.
  \begin{dfn}\quad
    \begin{enumerate}
      \item A \emph{valuation} in a dynamic locale $\mathcal{X} = (X, \le, \supset, \rightarrow, \wedge, \vee, \nabla, 0, 1)$ with canonical implication $\rightarrow$, is a function $v$ which takes all formulas in $\L$ ($\L_*$) to $X$, where $v(\bot) = 0$, $v(\top) = 1$, $v(\nabla A) = \nabla v(A)$ and  $v(A \circ B) = v(A) \circ v(B)$ for $\circ \in \{\vee, \wedge, \rightarrow, \supset\}$.
      \item We say that a sequent $\Gamma \Rightarrow \Delta$ is \emph{true} for a valuation $V$ in a dynamic locale $\mathcal{X}$, written $(\mathcal{X}, v) \vDash \Gamma \Rightarrow \Delta$, if $\bigwedge_{A \in \Gamma} v(A) \le v(\Delta)$, where $v(\Delta) = 0$ if $\Delta = \emptyset$. Truth of a formula $A$ is defined as the truth of the sequent $\Rightarrow A$.
      \item A sequent $\Gamma \Rightarrow \Delta$ is \emph{valid} if for all dynamic locales $\mathcal{X}$ and all valuations $v$, we have $(\mathcal{X}, v) \vDash \Gamma \Rightarrow \Delta$.
    \end{enumerate}
  \end{dfn}
  
  The logic of dynamic locales is defined by the following system, which we call $\GSTN$ here.
  The system $\GSTN$ is defined over the language $\L$ consisting of all the above rules. Moreover, the system $\GSTN_*$ is defined over $\L_*$ consisting of all the above rules except $L \supset$ and $R \supset$.

\begin{multicols}{3}
  \begin{prooftree}
    \AXC{}
    \RightLabel{$Id$}
    \UIC{$A \Rightarrow A$}
  \end{prooftree}
  \columnbreak
  \begin{prooftree}
    \AXC{}
    \RightLabel{$L \bot$}
    \UIC{$ \bot \Rightarrow $}		
  \end{prooftree}
  \columnbreak
  \begin{prooftree}
    \AXC{}
    \RightLabel{$R \top$}
    \UIC{$ \Rightarrow \top$}
  \end{prooftree}
\end{multicols}

\begin{multicols}{3}
  \begin{prooftree}
    \AXC{$ \Gamma \Rightarrow \Delta$}
    \RightLabel{$L w$}
    \UIC{$ \Gamma, {A} \Rightarrow \Delta$}
  \end{prooftree}
  \columnbreak
  \begin{prooftree}
    \AXC{$ \Gamma \Rightarrow$}
    \RightLabel{$R w$}
     \UIC{$\Gamma \Rightarrow {A}$}		
  \end{prooftree}
  \columnbreak
   \begin{prooftree}
     \AXC{$ \Gamma, A, A \Rightarrow \Delta$}
     \RightLabel{$Lc$}
     \UIC{$\Gamma, {A} \Rightarrow \Delta$}		
   \end{prooftree}
 \end{multicols}
 
 \begin{prooftree}
   \AXC{$\Gamma \Rightarrow A$}
   \AXC{$\Sigma, A \Rightarrow \Delta$}
   \RightLabel{$cut$}
   \BIC{$\Gamma, \Sigma \Rightarrow \Delta$}
 \end{prooftree}

\begin{multicols}{3}
  \begin{prooftree}
    \AXC{$ \Gamma, A \Rightarrow \Delta$}
    \RightLabel{$L \wedge_1$}
    \UIC{$ \Gamma, {A \wedge B} \Rightarrow \Delta$}		
  \end{prooftree}
  \columnbreak
  \begin{prooftree}
    \AXC{$ \Gamma, B \Rightarrow \Delta$}
    \RightLabel{$L \wedge_2$}
    \UIC{$\Gamma, {A \wedge B} \Rightarrow \Delta$}		
  \end{prooftree}
  \columnbreak
  \begin{prooftree}
    \AXC{$\Gamma \Rightarrow A$}
    \AXC{$\Gamma \Rightarrow B$}
    \RightLabel{$R \wedge$}
    \BIC{$ \Gamma \Rightarrow {A \wedge B}$}
  \end{prooftree}
\end{multicols}

\begin{prooftree}
  \AXC{$ \Gamma, A \Rightarrow \Delta$}
  \AXC{$\Gamma, B \Rightarrow \Delta$}
  \RightLabel{$L \vee$}
  \BIC{$ \Gamma, {A \vee B} \Rightarrow \Delta$}
\end{prooftree}


\begin{multicols}{2}
  \columnbreak
  \begin{prooftree}
    \AXC{$\Gamma \Rightarrow A$}
    \RightLabel{$R \vee_1$}
    \UIC{$\Gamma \Rightarrow {A \vee B}$}		
  \end{prooftree}
  \columnbreak
  \begin{prooftree}
    \AXC{$\Gamma \Rightarrow B$}
    \RightLabel{$R \vee_2$}
    \UIC{$\Gamma \Rightarrow {A \vee B}$}		
  \end{prooftree}
\end{multicols}

 \begin{multicols}{2}
  \begin{prooftree}
    \AXC{$\Gamma \Rightarrow A$}
    \AXC{$\Gamma, B \Rightarrow \Delta$}
    \RightLabel{$L \rightarrow$}
    \BIC{$\Gamma, {\nabla (A \rightarrow B)} \Rightarrow \Delta$}		
  \end{prooftree}
  \columnbreak
  \begin{prooftree}
    \AXC{$\nabla \Gamma, A \Rightarrow B$}
    \RightLabel{$R \rightarrow$}
    \UIC{$\Gamma \Rightarrow {A \rightarrow B}$}		
  \end{prooftree}
\end{multicols}

\begin{multicols}{2}
  \begin{prooftree}
    \AXC{$\Gamma \Rightarrow A$}
    \AXC{$\Gamma, B \Rightarrow \Delta$}
    \RightLabel{$L \supset$}
    \BIC{$\Gamma, A \supset B \Rightarrow \Delta$}		
  \end{prooftree}
  \columnbreak
  \begin{prooftree}
    \AXC{$\Gamma, A \Rightarrow B$}
    \RightLabel{$R \supset$}
    \UIC{$\Gamma \Rightarrow A \supset B$}		
  \end{prooftree}
\end{multicols}

\begin{prooftree}
  \AXC{$\Gamma \Rightarrow \Delta$}
  \RightLabel{$N$}
  \UIC{$\nabla \Gamma \Rightarrow \nabla \Delta$}
\end{prooftree}


We refer to the rules $Id$, $L \bot$ and $R \top$ as the \emph{axioms}, $L \wedge$, $R \wedge$, $L \vee$, $R \vee_1$, $R \vee_2$, $L \rightarrow$, $R \rightarrow$, $L \supset$ and $R \supset$ as the \emph{logical rules}, and $Lw$ and $Rw$ as the \emph{structural rules}.

  The logic of dynamic locales, $\ldl_*$, is the logic of the system $\GSTN_*$. We also denote the logic of the extended system $\GSTN$ by $\ldl$.
  
  It is shown in \cite{amir} that $\ldl$ and $\ldl_*$ are sound and complete with respect to the class of all dynamic locales.
  \begin{thm} \quad
    \begin{enumerate}
      \item For any formula $A$ in the extended language $\L$, $\ldl \vdash A$ iff $A$ is valid in all dynamic locales.
      \item For any formula $A$ in the language $\L_*$, $\ldl_* \vdash A$ iff $A$ is valid in all dynamic locales.
    \end{enumerate}
  \end{thm}
  
  It is easily seen that the above system $\GSTN$ is not analytic, as the rule $R \rightarrow$ introduces a $\nabla$ in its premise, and the rule $cut$ introduces a new formula. The added $\nabla$ in the premise of the rule $R \rightarrow$ can be neglected by relaxing the analyticity criterion up to $\nabla$, which would be enough for proof theoretic methods to be applicable. But we need a system without $cut$.  One could observe that the $cut$ rule can not be eliminated from the system above, using common methods for cut-elimination, i.e. induction on the height of the proof-tree. So, as we will see in the next section, we would need to drastically change many rules of this system, to reach an equivalent system without the $cut$ rule.

\section{An analytic system for $\ldl$ and $\ldl_*$}
The systems $\LDL$ and $\LDL_*$ are defined over the language $\L$ and $\L_*$, respectively. The system $\LDL$ is defined by the table of rules below, while the system $\LDL_*$ is defined by the same rules, except $L \supset$ and $R \supset$.
\documentclass[10pt,a4paper]{amsart}
\input{preamble}

\begin{document}

\title{Interpolation for the Logic of Dynamic Locales}
\author{Amirhossein Akbar Tabatabai}


\author{Majid Alizadeh}

\author{Alireza Mahmoudian}

\date{\today}
 
\begin{abstract}
	\input{abstract}
\end{abstract}

\maketitle

\keywords{Cut-elimination, Interpolation, Dynamics in general topological spaces}

\subjclass{03F05, 03C40, 37B02}

\section{Introduction}
\input{introduction}

\section{Preliminaries}
\input{preliminaries}

\section{An analytic system for $\ldl$ and $\ldl_*$}
\input{system}

\bibliographystyle{abbrv}
\bibliography{refs}

\end{document}

\begin{rem}\label{rem:ldl-gstn}
  Notice the differences between $\GSTN$ and $\LDL$. The latter lacks the rules $cut$ and $Lc$, its $Id ^p$ rule is just $Id$, limited to the atomic formulas, and its $LW$ can introduce more than one formula to its conclusion, which is admissible in $\GSTN$ by multiple instances of $Lw$. It also differs in $L \wedge ^n$, $L \vee ^n$, $L \rightarrow ^n$ and $L \supset ^n$, which also have arbitrary number of $\nabla$s on their principal formula, which is distributed over its subformulas in their premises.  
\end{rem}

In the rest of this section we are going to show that $\LDL$ and $\GSTN$ are equivalent, i.e. prove exactly the same set of sequents. Hence, proving that the logic $\ldl$ has an analytic system. To reach this aim, we have to show the admissibility of $Id$, $Lc$ and $cut$ from $\GSTN$ in $\LDL$, and the admissibility of $L \wedge ^n$, $L \vee ^n$, $L \rightarrow ^n$ and $L \supset ^n$ in $\GSTN$. We also need some lemmas which facilitate our admissibility proofs.

\begin{lem}\label{lem:l-nabla-dist} For each $n \geq 0$
	\begin{enumerate}
		\item $\GSTN \vdash \nabla^n (A \vee B) \Rightarrow \nabla^n A \vee \nabla^n B$.
	
		\item $\GSTN \vdash \nabla^n (A \wedge B) \Rightarrow \nabla^n A \wedge \nabla^n B$. 
	
		\item $\GSTN \vdash \nabla^n (A \rightarrow B) \Rightarrow \nabla^n A \rightarrow \nabla^n B$.
		
    \item $\GSTN \vdash \nabla^n (A \supset B) \Rightarrow \nabla^n A \supset \nabla^n B$.
	\end{enumerate}
\end{lem}
\begin{proof}

  For (1), first, observe that

  \begin{prooftree}
    \AXC{}
    \RightLabel{$R \top$}
    \UIC{$\Rightarrow \top$}
    \RightLabel{$Lw$}
    \UIC{$\nabla (\top \rightarrow C) \Rightarrow \top$}
  
    \AXC{}
    \RightLabel{$Id$}
    \UIC{$C \Rightarrow C$}
    \RightLabel{$Lw$}
    \UIC{$\nabla (\top \rightarrow C), C \Rightarrow C$}
  
    \RightLabel{$L \rightarrow$}
    \BIC{$\nabla (\top \rightarrow C) \Rightarrow C$}
  \end{prooftree}
  
  
  Now, let $\D$ be the proof-tree above with $C = \nabla A \vee \nabla B$.
  
  \begin{prooftree}
    \AXC{}
    \RightLabel{$Id$}
    \UIC{$\nabla A \Rightarrow \nabla A$}
    \RightLabel{$R\vee_1$}
    \UIC{$\nabla A \Rightarrow \nabla A \vee \nabla B$}
    \RightLabel{$Lw$}
    \UIC{$\nabla A , \top \Rightarrow \nabla A \vee \nabla B$}
    \RightLabel{$R \rightarrow$}
    \UIC{$A \Rightarrow \top \rightarrow (\nabla A \vee \nabla B)$}
  
    \AXC{}
    \RightLabel{$Id$}
    \UIC{$\nabla B \Rightarrow \nabla B$}
    \RightLabel{$R\vee_2$}
    \UIC{$\nabla B \Rightarrow \nabla A \vee \nabla B$}
    \RightLabel{$Lw$}
    \UIC{$\nabla B , \top \Rightarrow \nabla A \vee \nabla B$}
    \RightLabel{$R \rightarrow$}
    \UIC{$B \Rightarrow \top \rightarrow (\nabla A \vee \nabla B)$}
  
    \RightLabel{$L\vee$}
    \BIC{$A \vee B \Rightarrow \top \rightarrow (\nabla A \vee \nabla B)$}
    \RightLabel{$N$}
    \UIC{$\nabla (A \vee B) \Rightarrow \nabla (\top \rightarrow (\nabla A \vee \nabla B))$}
  
    \AXC{$\D$}
  
    \RightLabel{$cut$}
    \BIC{$\nabla (A \vee B) \Rightarrow \nabla A \vee \nabla B$}
  \end{prooftree}
  
  Call this proof-tree $\D_1$. Construct $\D_{n+1}$ inductively as follows.
  
  \begin{prooftree}
    \AXC{$\D_n$}
    \noLine
    \UIC{$\nabla^n (A \vee B) \Rightarrow \nabla^n A \vee \nabla^n B$}
    \RightLabel{$N$}
    \UIC{$\nabla^{n+1} (A \vee B) \Rightarrow \nabla (\nabla^n A \vee \nabla^n B)$}
  
    \AXC{$\D_1$}
    \noLine
    \UIC{$\nabla (\nabla^n A \vee \nabla^n B) \Rightarrow \nabla^{n+1} A \vee \nabla^{n+1} B$}
    
    \RightLabel{$cut$} \LeftLabel{$\D_{n+1}:$}
    \BIC{$\nabla^{n+1} (A \vee B) \Rightarrow \nabla^{n+1} A \vee \nabla^{n+1} B$}
  \end{prooftree}


  For (2), (3) and (4), consider the following proof-trees:
  \begin{prooftree}
    \AXC{}
    \RightLabel{$Id$}
    \UIC{$A \Rightarrow A$}
    \RightLabel{$L \wedge_1$}
    \UIC{$A \wedge B \Rightarrow A$}
    \RightLabel{$N^{(*)}$} \doubleLine
    \UIC{$\nabla^n (A \wedge B) \Rightarrow \nabla^n A$}
  
    \AXC{}
    \RightLabel{$Id$}
    \UIC{$B \Rightarrow B$}
    \RightLabel{$L \wedge_2$}
    \UIC{$A \wedge B \Rightarrow B$}
    \RightLabel{$N^{(*)}$} \doubleLine	
    \UIC{$\nabla^n (A \wedge B) \Rightarrow \nabla^n B$}
    
    \RightLabel{$R \wedge$}
    \BIC{$\nabla^n (A \wedge B) \Rightarrow \nabla^n A \wedge \nabla^n B$}
  \end{prooftree}

  \begin{prooftree}
    \AXC{}
    \RightLabel{$Id$}
    \UIC{$A \Rightarrow A$}
    \RightLabel{$Lw$}
    \UIC{$A, \nabla (A \rightarrow B) \Rightarrow A$}
  
    \AXC{}
    \RightLabel{$Id$}
    \UIC{$B \Rightarrow B$}
    \RightLabel{$Lw$}
    \UIC{$A, \nabla (A \rightarrow B), B \Rightarrow B$}
  
    \RightLabel{$L \rightarrow$}
    \BIC{$\nabla (A \rightarrow B) , A \Rightarrow B$}
    \RightLabel{$N^{(*)}$} \doubleLine
    \UIC{$\nabla^{n+1} (A \rightarrow B) , \nabla^n A \Rightarrow \nabla^n B$}
    \RightLabel{$R \rightarrow$}
    \UIC{$\nabla^n (A \rightarrow B) \Rightarrow \nabla^n A \rightarrow \nabla^n B$}
  \end{prooftree}


  \begin{prooftree}
    \AXC{}
    \RightLabel{$Id$}
    \UIC{$A \Rightarrow A$}
    \RightLabel{$Lw$}
    \UIC{$A, A \supset B \Rightarrow A$}
  
    \AXC{}
    \RightLabel{$Id$}
    \UIC{$B \Rightarrow B$}
    \RightLabel{$Lw$}
    \UIC{$A, A \supset B, B \Rightarrow B$}
  
    \RightLabel{$L \supset$}
    \BIC{$A \supset B , A \Rightarrow B$}
    \RightLabel{$N^{(*)}$} \doubleLine
    \UIC{$\nabla^n (A \supset B) , \nabla^n A \Rightarrow \nabla^n B$}
    \RightLabel{$R \supset$}
    \UIC{$\nabla^n (A \supset B) \Rightarrow \nabla^n A \supset \nabla^n B$}
  \end{prooftree}

  where $N^{(*)}$ denotes multiple uses of the rule $N$.
\end{proof}


\begin{thm}\label{thm:id-adm}
	$\LDL \vdash A \Rightarrow A$.
\end{thm}
\begin{proof}
	By induction on the structure of $A$. If $A$ is an atom, $\bot$ or $\top$, we would have $A \Rightarrow A$ using the axioms of $\LDL$, $Lw$ and $Rw$. If $A = B \circ C$, where $\circ \in \{\wedge, \vee, \supset\}$, the claim is clear from the induction hypothesis for $B$ and $C$. We will explain the proof in details for $A = B \rightarrow C$ and $A = \nabla B$. Notice that the subtrees denoted by IH are obtained from the induction hypothesis.

($B \rightarrow C$)
\begin{prooftree}
  \AXC{IH} \noLine
  \UIC{$B \Rightarrow B$} \RightLabel{$Lw$}
  \UIC{$\nabla (B \rightarrow C), B \Rightarrow B$}
  \AXC{IH} \noLine
  \UIC{$C \Rightarrow C$} \RightLabel{$Lw$}
  \UIC{$\nabla (B \rightarrow C), C \Rightarrow C$}
  \RightLabel{$L \rightarrow$}
  \BIC{$\nabla (B \rightarrow C), B \Rightarrow C$}
  \RightLabel{$R \rightarrow$}
  \UIC{$B \rightarrow C \Rightarrow B \rightarrow C$}
\end{prooftree}

($\nabla B$)
\begin{prooftree}
  \AXC{IH} \noLine
  \UIC{$B \Rightarrow B$}
  \RightLabel{$N$}
  \UIC{$\nabla B \Rightarrow \nabla B$}
\end{prooftree}
\end{proof}

In order to prove the admissibility of $Lc$, we also need the following \emph{inversion} lemmas for some of the rules of $\LDL$.

\begin{lem}[Inversion]\label{lem:inv} \quad
	\begin{enumerate}
		\item If $\LDL \vdash_h \Gamma, \nabla^n (A \wedge B) \Rightarrow \Delta$ then $\LDL \vdash_h \Gamma, \nabla^n A, \nabla^n B \Rightarrow \Delta$.
		\item If $\LDL \vdash_h \Gamma, \nabla^n (A \vee B) \Rightarrow \Delta$ then $\LDL \vdash_h \Gamma, \nabla^n A \Rightarrow \Delta$ and $\LDL \vdash_h \Gamma, \nabla^n B \Rightarrow \Delta$.
  	\item If $\LDL \vdash_h \Gamma, \nabla^n (A \supset B) \Rightarrow \Delta$ then $\LDL \vdash_h \Gamma, \nabla^n B \Rightarrow \Delta$.
	\end{enumerate}
\end{lem}
\begin{proof}
  We only explain the proof for (3), the other two parts are similar. We use induction on $h$, the height of the proof-tree for $\Gamma, \nabla^n (A \supset B) \Rightarrow \Delta$. It is clear that $h$ can not be $0$, since $\Gamma, \nabla^n (A \supset B) \Rightarrow \Delta$ can not be proved by an axiom. If the proof-tree ends with $L \supset ^n$, and $\nabla^n (A \supset B)$ is the principal formula, then the claim would be obvious. In any other case, just commute the last rule with the sequent obtained from the induction hypothesis. Here we explain the case for $N$.

  Suppose the last rule is $N$. So we have
  \begin{prooftree}
    \AXC{$\Gamma, \nabla^n (A \supset B) \Rightarrow \Delta$} \RightLabel{$N$}
    \UIC{$\nabla \Gamma, \nabla^{n+1} (A \supset B) \Rightarrow \nabla \Delta$}
  \end{prooftree}
  at the end of a proof-tree of height $h > 0$. From the induction hypothesis for the premise, we have a proof for $\Gamma, \nabla^n B \Rightarrow \Delta$ of heigt $h - 1$. Using $N$, we have a proof-tree for $\nabla \Gamma, \nabla^{n+1} B \Rightarrow \nabla \Delta$ with height $h$.

  The cases for the other rules are similar.
\end{proof}

The next theorem shows that the rule $Lc$ from $\GSTN$ is admissible in $\LDL$.
\begin{thm}[Height-preserving admissibility of $Lc$]\label{thm:lc-adm} For any sequent $\Gamma$, $\Delta$ and $A$, and any $h \geq 0$, if $\LDL \vdash_h \Gamma, A, A \Rightarrow \Delta$ then $\LDL \vdash_h \Gamma, A \Rightarrow \Delta$.
\end{thm}
\begin{proof}
  By induction on $h$. For $h = 0$, the proof for $\Gamma, A, A \Rightarrow \Delta$ must be an axiom, which is impossible. For $h > 0$, we investigate different cases by the last rule of the proof of $\Gamma, A, A \Rightarrow \Delta$. The cases where none of occurrences of $A$ are principal are trivial, as it is enough to apply the same rule on the proofs the induction hypothesis provides.
  
  If one occurrence of $A$ is principal, we have the following cases based on the last rule of the proof:
	\begin{itemize}
		\item[$(L \wedge)$] If the last rule is $L \wedge$, then $A = \nabla^n (B \wedge C)$ it is in the form:

		\begin{prooftree}
			\AXC{$\D_0$} \noLine
			\UIC{$\Gamma, \nabla^n B, \nabla^n C, \nabla^n (B \wedge C) \Rightarrow \Delta$}
			\RightLabel{$L \wedge$}
			\UIC{$\Gamma, \nabla^n (B \wedge C), \nabla^n (B \wedge C) \Rightarrow \Delta$}		
		\end{prooftree}

		Then, by Lemma \ref{lem:inv} we have a proof for $\Gamma, \nabla^n B, \nabla^n B, \nabla^n C, \nabla^n C \Rightarrow \Delta$ of height less than $h$. Using induction hypothesis twice, we would have a proof for $\Gamma, \nabla^n B, \nabla^n C \Rightarrow \Delta$ of height less than $h$. Finally, $L \wedge$ will result in the desired proof with height at most $h$.
	
		\item[$(L \vee)$] If the last rule is $L \vee$, then $A = \nabla^n (B \vee C)$ and the last rule is in the form:
		\begin{prooftree}
			\AXC{$\D_0$} \noLine
			\UIC{$ \Gamma, \nabla^n B, \nabla^n (B \vee C) \Rightarrow \Delta$}
			\AXC{$\D_1$} \noLine
			\UIC{$\Gamma, \nabla^n C, \nabla^n (B \vee C) \Rightarrow \Delta$}
			\RightLabel{$L \vee$}
			\BIC{$ \Gamma, \nabla^n (B \vee C), \nabla^n (B \vee C) \Rightarrow \Delta$}		
		\end{prooftree}
		By Lemma \ref{lem:inv} for $\D_0$ and $\D_1$, we have proofs for $\Gamma, \nabla^n B, \nabla^n B \Rightarrow \Delta$ and $\Gamma, \nabla^n C, \nabla^n C \Rightarrow \Delta$ of heights less than $h$. By induction hypothesis, we have proofs for $\Gamma, \nabla^n B \Rightarrow \Delta$ and $\Gamma, \nabla^n C \Rightarrow \Delta$ of heights less than $h$. By $L \vee$ we have the desired proof with height at most $h$.
	
		\item[$(L \rightarrow)$] If the last rule is $L \rightarrow$, then $A = \nabla^{n+1} (B \rightarrow C)$ and the last rule is in the form:
		\begin{prooftree}
			\AXC{$\D_0$} \noLine
			\UIC{$\Gamma, \nabla^{n+1} (B \rightarrow C), \nabla^{n+1} (B \rightarrow C) \Rightarrow \nabla^n B$}
			\AXC{$\D_1$} \noLine
			\UIC{$\Gamma, \nabla^n C, (\nabla^{n+1} (B \rightarrow C)), \nabla^{n+1} (B \rightarrow C) \Rightarrow \Delta$}
			\RightLabel{$L \rightarrow$}
			\BIC{$\Gamma, \nabla^{n+1} (B \rightarrow C), \nabla^{n+1} (B \rightarrow C) \Rightarrow \Delta$}
		\end{prooftree}
		By induction hypothesis, we have a proof for $\Gamma, \nabla^{n+1} (B \rightarrow C) \Rightarrow \nabla^n B$ and $\Gamma, \nabla^n C, \nabla^{n+1} (B \rightarrow C) \Rightarrow \Delta$ with height less than $h$. Finally, by $L \rightarrow$ we have the desired proof with height at most $h$.

	
		\item[$(L \supset)$] If the last rule is $L \supset$, then $A = \nabla^n (B \supset C)$ and the last rule is in the form:
		\begin{prooftree}
			\AXC{$\D_0$} \noLine
			\UIC{$\Gamma, \nabla^n (B \supset C), \nabla^n (B \supset C) \Rightarrow \nabla^n B$}
			\AXC{$\D_1$} \noLine
			\UIC{$\Gamma, \nabla^n C, \nabla^n (B \supset C) \Rightarrow \Delta$}
			\RightLabel{$L \supset$}
			\BIC{$\Gamma, \nabla^n (B \supset C), \nabla^n (B \supset C) \Rightarrow \Delta$}
		\end{prooftree}
		By Lemma \ref{lem:inv} for $\D_1$, we have a proof for $\Gamma, \nabla^n C, \nabla^n C \Rightarrow \Delta$ of height less than $h$. By induction hypothesis, we have proofs for $\Gamma, \nabla^n (B \supset C) \Rightarrow \nabla^n B$ and $\Gamma, \nabla^n C \Rightarrow \Delta$, both with heights less than $h$. Finally, by $L \supset$ we have the desired proof, with height at most $h$.
	\end{itemize}
\end{proof}

It remains to show that $cut$ is admissible in $\LDL$.

\begin{thm}\label{thm:cut-adm}
  If $\LDL \vdash \Gamma \Rightarrow A$ and $\LDL \vdash \Sigma, A \Rightarrow \Delta$, then $\LDL \vdash \Gamma, \Sigma \Rightarrow \Delta$.
\end{thm}

We postpone the proof of Theorem \ref{thm:cut-adm} to the next section. Nevertheless, we will use its result to show that $\LDL$ and $\GSTN$ are equivalent, and therefore, share the same logic $\ldl$.

\begin{nota}
  Thanks to Theorems \ref{thm:id-adm}, \ref{thm:lc-adm} and \ref{thm:cut-adm}, from now on, we pretend that the rules $Id$, $Lc$ and $cut$ are available in $\LDL$ as well. Therefore, we use them in the proof-trees in $\LDL$ without any further explanation.
\end{nota}

Now, we are ready to prove the equivalence of two systems.

\begin{thm}\label{thm:ldl-eq-gstn}
  Any sequent $\Gamma \Rightarrow \Delta$ in the language $\L$ is provable in $\GSTN$ iff it is provable in $\LDL$.
\end{thm}
\begin{proof}
  By induction on the proof-tree for $\Gamma \Rightarrow \Delta$, the claim is easy to observe in both directions. For the left to right direction, as was observed in Remark \ref{rem:ldl-gstn}, it suffices to use Theorems \ref{thm:id-adm}, \ref{thm:lc-adm} and \ref{thm:cut-adm}. For the other direction, use Lemma \ref{lem:l-nabla-dist}. We will explain the case where the $\LDL$ proof-tree for $\Gamma \Rightarrow \Delta$ ends with $L \rightarrow ^n$. In this case, by inductin hypothesis, we have proof-trees for $\Gamma, \nabla^{n+1} (A \rightarrow B) \Rightarrow \nabla^n A$ and $\Gamma, \nabla^{n+1} (A \rightarrow B), \nabla^n B \Rightarrow \Delta$ in $\GSTN$. We also have $\nabla^n (A \rightarrow B) \Rightarrow \nabla^n A \rightarrow \nabla^n B$ from Lemma \ref{lem:l-nabla-dist} part (3).

  \begin{prooftree}
    \AXC{$\nabla^n (A \rightarrow B) \Rightarrow \nabla^n A \rightarrow \nabla^n B$}
    \RightLabel{$N$}
    \UIC{$\nabla^{n+1} (A \rightarrow B) \Rightarrow \nabla (\nabla^n A \rightarrow \nabla^n B)$}

    \AXC{$\Gamma, \nabla^{n+1} (A \rightarrow B) \Rightarrow \nabla^n A$}
    \AXC{$\Gamma, \nabla^{n+1} (A \rightarrow B), \nabla^n B \Rightarrow \Delta$}
    \RightLabel{$R \rightarrow$}
    \BIC{$\Gamma, \nabla^{n+1} (A \rightarrow B), \nabla (\nabla^n A \rightarrow \nabla^n B) \Rightarrow \Delta$}

    \RightLabel{$cut$}
    \BIC{$\Gamma, (\nabla^{n+1} (A \rightarrow B))^2 \Rightarrow \Delta$}
    \RightLabel{$Lc$}
    \UIC{$\Gamma, \nabla^{n+1} (A \rightarrow B) \Rightarrow \Delta$}
  \end{prooftree}

  Other cases are similar.
\end{proof}

\begin{cor}
  The logic of dynamic locales $\ldl$ is the logic of the system $\LDL$.
\end{cor}

\bibliographystyle{abbrv}
\bibliography{refs}

\end{document}

\begin{rem}\label{rem:ldl-gstn}
  Notice the differences between $\GSTN$ and $\LDL$. The latter lacks the rules $cut$ and $Lc$, its $Id ^p$ rule is just $Id$, limited to the atomic formulas, and its $LW$ can introduce more than one formula to its conclusion, which is admissible in $\GSTN$ by multiple instances of $Lw$. It also differs in $L \wedge ^n$, $L \vee ^n$, $L \rightarrow ^n$ and $L \supset ^n$, which also have arbitrary number of $\nabla$s on their principal formula, which is distributed over its subformulas in their premises.  
\end{rem}

In the rest of this section we are going to show that $\LDL$ and $\GSTN$ are equivalent, i.e. prove exactly the same set of sequents. Hence, proving that the logic $\ldl$ has an analytic system. To reach this aim, we have to show the admissibility of $Id$, $Lc$ and $cut$ from $\GSTN$ in $\LDL$, and the admissibility of $L \wedge ^n$, $L \vee ^n$, $L \rightarrow ^n$ and $L \supset ^n$ in $\GSTN$. We also need some lemmas which facilitate our admissibility proofs.

\begin{lem}\label{lem:l-nabla-dist} For each $n \geq 0$
	\begin{enumerate}
		\item $\GSTN \vdash \nabla^n (A \vee B) \Rightarrow \nabla^n A \vee \nabla^n B$.
	
		\item $\GSTN \vdash \nabla^n (A \wedge B) \Rightarrow \nabla^n A \wedge \nabla^n B$. 
	
		\item $\GSTN \vdash \nabla^n (A \rightarrow B) \Rightarrow \nabla^n A \rightarrow \nabla^n B$.
		
    \item $\GSTN \vdash \nabla^n (A \supset B) \Rightarrow \nabla^n A \supset \nabla^n B$.
	\end{enumerate}
\end{lem}
\begin{proof}

  For (1), first, observe that

  \begin{prooftree}
    \AXC{}
    \RightLabel{$R \top$}
    \UIC{$\Rightarrow \top$}
    \RightLabel{$Lw$}
    \UIC{$\nabla (\top \rightarrow C) \Rightarrow \top$}
  
    \AXC{}
    \RightLabel{$Id$}
    \UIC{$C \Rightarrow C$}
    \RightLabel{$Lw$}
    \UIC{$\nabla (\top \rightarrow C), C \Rightarrow C$}
  
    \RightLabel{$L \rightarrow$}
    \BIC{$\nabla (\top \rightarrow C) \Rightarrow C$}
  \end{prooftree}
  
  
  Now, let $\D$ be the proof-tree above with $C = \nabla A \vee \nabla B$.
  
  \begin{prooftree}
    \AXC{}
    \RightLabel{$Id$}
    \UIC{$\nabla A \Rightarrow \nabla A$}
    \RightLabel{$R\vee_1$}
    \UIC{$\nabla A \Rightarrow \nabla A \vee \nabla B$}
    \RightLabel{$Lw$}
    \UIC{$\nabla A , \top \Rightarrow \nabla A \vee \nabla B$}
    \RightLabel{$R \rightarrow$}
    \UIC{$A \Rightarrow \top \rightarrow (\nabla A \vee \nabla B)$}
  
    \AXC{}
    \RightLabel{$Id$}
    \UIC{$\nabla B \Rightarrow \nabla B$}
    \RightLabel{$R\vee_2$}
    \UIC{$\nabla B \Rightarrow \nabla A \vee \nabla B$}
    \RightLabel{$Lw$}
    \UIC{$\nabla B , \top \Rightarrow \nabla A \vee \nabla B$}
    \RightLabel{$R \rightarrow$}
    \UIC{$B \Rightarrow \top \rightarrow (\nabla A \vee \nabla B)$}
  
    \RightLabel{$L\vee$}
    \BIC{$A \vee B \Rightarrow \top \rightarrow (\nabla A \vee \nabla B)$}
    \RightLabel{$N$}
    \UIC{$\nabla (A \vee B) \Rightarrow \nabla (\top \rightarrow (\nabla A \vee \nabla B))$}
  
    \AXC{$\D$}
  
    \RightLabel{$cut$}
    \BIC{$\nabla (A \vee B) \Rightarrow \nabla A \vee \nabla B$}
  \end{prooftree}
  
  Call this proof-tree $\D_1$. Construct $\D_{n+1}$ inductively as follows.
  
  \begin{prooftree}
    \AXC{$\D_n$}
    \noLine
    \UIC{$\nabla^n (A \vee B) \Rightarrow \nabla^n A \vee \nabla^n B$}
    \RightLabel{$N$}
    \UIC{$\nabla^{n+1} (A \vee B) \Rightarrow \nabla (\nabla^n A \vee \nabla^n B)$}
  
    \AXC{$\D_1$}
    \noLine
    \UIC{$\nabla (\nabla^n A \vee \nabla^n B) \Rightarrow \nabla^{n+1} A \vee \nabla^{n+1} B$}
    
    \RightLabel{$cut$} \LeftLabel{$\D_{n+1}:$}
    \BIC{$\nabla^{n+1} (A \vee B) \Rightarrow \nabla^{n+1} A \vee \nabla^{n+1} B$}
  \end{prooftree}


  For (2), (3) and (4), consider the following proof-trees:
  \begin{prooftree}
    \AXC{}
    \RightLabel{$Id$}
    \UIC{$A \Rightarrow A$}
    \RightLabel{$L \wedge_1$}
    \UIC{$A \wedge B \Rightarrow A$}
    \RightLabel{$N^{(*)}$} \doubleLine
    \UIC{$\nabla^n (A \wedge B) \Rightarrow \nabla^n A$}
  
    \AXC{}
    \RightLabel{$Id$}
    \UIC{$B \Rightarrow B$}
    \RightLabel{$L \wedge_2$}
    \UIC{$A \wedge B \Rightarrow B$}
    \RightLabel{$N^{(*)}$} \doubleLine	
    \UIC{$\nabla^n (A \wedge B) \Rightarrow \nabla^n B$}
    
    \RightLabel{$R \wedge$}
    \BIC{$\nabla^n (A \wedge B) \Rightarrow \nabla^n A \wedge \nabla^n B$}
  \end{prooftree}

  \begin{prooftree}
    \AXC{}
    \RightLabel{$Id$}
    \UIC{$A \Rightarrow A$}
    \RightLabel{$Lw$}
    \UIC{$A, \nabla (A \rightarrow B) \Rightarrow A$}
  
    \AXC{}
    \RightLabel{$Id$}
    \UIC{$B \Rightarrow B$}
    \RightLabel{$Lw$}
    \UIC{$A, \nabla (A \rightarrow B), B \Rightarrow B$}
  
    \RightLabel{$L \rightarrow$}
    \BIC{$\nabla (A \rightarrow B) , A \Rightarrow B$}
    \RightLabel{$N^{(*)}$} \doubleLine
    \UIC{$\nabla^{n+1} (A \rightarrow B) , \nabla^n A \Rightarrow \nabla^n B$}
    \RightLabel{$R \rightarrow$}
    \UIC{$\nabla^n (A \rightarrow B) \Rightarrow \nabla^n A \rightarrow \nabla^n B$}
  \end{prooftree}


  \begin{prooftree}
    \AXC{}
    \RightLabel{$Id$}
    \UIC{$A \Rightarrow A$}
    \RightLabel{$Lw$}
    \UIC{$A, A \supset B \Rightarrow A$}
  
    \AXC{}
    \RightLabel{$Id$}
    \UIC{$B \Rightarrow B$}
    \RightLabel{$Lw$}
    \UIC{$A, A \supset B, B \Rightarrow B$}
  
    \RightLabel{$L \supset$}
    \BIC{$A \supset B , A \Rightarrow B$}
    \RightLabel{$N^{(*)}$} \doubleLine
    \UIC{$\nabla^n (A \supset B) , \nabla^n A \Rightarrow \nabla^n B$}
    \RightLabel{$R \supset$}
    \UIC{$\nabla^n (A \supset B) \Rightarrow \nabla^n A \supset \nabla^n B$}
  \end{prooftree}

  where $N^{(*)}$ denotes multiple uses of the rule $N$.
\end{proof}


\begin{thm}\label{thm:id-adm}
	$\LDL \vdash A \Rightarrow A$.
\end{thm}
\begin{proof}
	By induction on the structure of $A$. If $A$ is an atom, $\bot$ or $\top$, we would have $A \Rightarrow A$ using the axioms of $\LDL$, $Lw$ and $Rw$. If $A = B \circ C$, where $\circ \in \{\wedge, \vee, \supset\}$, the claim is clear from the induction hypothesis for $B$ and $C$. We will explain the proof in details for $A = B \rightarrow C$ and $A = \nabla B$. Notice that the subtrees denoted by IH are obtained from the induction hypothesis.

($B \rightarrow C$)
\begin{prooftree}
  \AXC{IH} \noLine
  \UIC{$B \Rightarrow B$} \RightLabel{$Lw$}
  \UIC{$\nabla (B \rightarrow C), B \Rightarrow B$}
  \AXC{IH} \noLine
  \UIC{$C \Rightarrow C$} \RightLabel{$Lw$}
  \UIC{$\nabla (B \rightarrow C), C \Rightarrow C$}
  \RightLabel{$L \rightarrow$}
  \BIC{$\nabla (B \rightarrow C), B \Rightarrow C$}
  \RightLabel{$R \rightarrow$}
  \UIC{$B \rightarrow C \Rightarrow B \rightarrow C$}
\end{prooftree}

($\nabla B$)
\begin{prooftree}
  \AXC{IH} \noLine
  \UIC{$B \Rightarrow B$}
  \RightLabel{$N$}
  \UIC{$\nabla B \Rightarrow \nabla B$}
\end{prooftree}
\end{proof}

In order to prove the admissibility of $Lc$, we also need the following \emph{inversion} lemmas for some of the rules of $\LDL$.

\begin{lem}[Inversion]\label{lem:inv} \quad
	\begin{enumerate}
		\item If $\LDL \vdash_h \Gamma, \nabla^n (A \wedge B) \Rightarrow \Delta$ then $\LDL \vdash_h \Gamma, \nabla^n A, \nabla^n B \Rightarrow \Delta$.
		\item If $\LDL \vdash_h \Gamma, \nabla^n (A \vee B) \Rightarrow \Delta$ then $\LDL \vdash_h \Gamma, \nabla^n A \Rightarrow \Delta$ and $\LDL \vdash_h \Gamma, \nabla^n B \Rightarrow \Delta$.
  	\item If $\LDL \vdash_h \Gamma, \nabla^n (A \supset B) \Rightarrow \Delta$ then $\LDL \vdash_h \Gamma, \nabla^n B \Rightarrow \Delta$.
	\end{enumerate}
\end{lem}
\begin{proof}
  We only explain the proof for (3), the other two parts are similar. We use induction on $h$, the height of the proof-tree for $\Gamma, \nabla^n (A \supset B) \Rightarrow \Delta$. It is clear that $h$ can not be $0$, since $\Gamma, \nabla^n (A \supset B) \Rightarrow \Delta$ can not be proved by an axiom. If the proof-tree ends with $L \supset ^n$, and $\nabla^n (A \supset B)$ is the principal formula, then the claim would be obvious. In any other case, just commute the last rule with the sequent obtained from the induction hypothesis. Here we explain the case for $N$.

  Suppose the last rule is $N$. So we have
  \begin{prooftree}
    \AXC{$\Gamma, \nabla^n (A \supset B) \Rightarrow \Delta$} \RightLabel{$N$}
    \UIC{$\nabla \Gamma, \nabla^{n+1} (A \supset B) \Rightarrow \nabla \Delta$}
  \end{prooftree}
  at the end of a proof-tree of height $h > 0$. From the induction hypothesis for the premise, we have a proof for $\Gamma, \nabla^n B \Rightarrow \Delta$ of heigt $h - 1$. Using $N$, we have a proof-tree for $\nabla \Gamma, \nabla^{n+1} B \Rightarrow \nabla \Delta$ with height $h$.

  The cases for the other rules are similar.
\end{proof}

The next theorem shows that the rule $Lc$ from $\GSTN$ is admissible in $\LDL$.
\begin{thm}[Height-preserving admissibility of $Lc$]\label{thm:lc-adm} For any sequent $\Gamma$, $\Delta$ and $A$, and any $h \geq 0$, if $\LDL \vdash_h \Gamma, A, A \Rightarrow \Delta$ then $\LDL \vdash_h \Gamma, A \Rightarrow \Delta$.
\end{thm}
\begin{proof}
  By induction on $h$. For $h = 0$, the proof for $\Gamma, A, A \Rightarrow \Delta$ must be an axiom, which is impossible. For $h > 0$, we investigate different cases by the last rule of the proof of $\Gamma, A, A \Rightarrow \Delta$. The cases where none of occurrences of $A$ are principal are trivial, as it is enough to apply the same rule on the proofs the induction hypothesis provides.
  
  If one occurrence of $A$ is principal, we have the following cases based on the last rule of the proof:
	\begin{itemize}
		\item[$(L \wedge)$] If the last rule is $L \wedge$, then $A = \nabla^n (B \wedge C)$ it is in the form:

		\begin{prooftree}
			\AXC{$\D_0$} \noLine
			\UIC{$\Gamma, \nabla^n B, \nabla^n C, \nabla^n (B \wedge C) \Rightarrow \Delta$}
			\RightLabel{$L \wedge$}
			\UIC{$\Gamma, \nabla^n (B \wedge C), \nabla^n (B \wedge C) \Rightarrow \Delta$}		
		\end{prooftree}

		Then, by Lemma \ref{lem:inv} we have a proof for $\Gamma, \nabla^n B, \nabla^n B, \nabla^n C, \nabla^n C \Rightarrow \Delta$ of height less than $h$. Using induction hypothesis twice, we would have a proof for $\Gamma, \nabla^n B, \nabla^n C \Rightarrow \Delta$ of height less than $h$. Finally, $L \wedge$ will result in the desired proof with height at most $h$.
	
		\item[$(L \vee)$] If the last rule is $L \vee$, then $A = \nabla^n (B \vee C)$ and the last rule is in the form:
		\begin{prooftree}
			\AXC{$\D_0$} \noLine
			\UIC{$ \Gamma, \nabla^n B, \nabla^n (B \vee C) \Rightarrow \Delta$}
			\AXC{$\D_1$} \noLine
			\UIC{$\Gamma, \nabla^n C, \nabla^n (B \vee C) \Rightarrow \Delta$}
			\RightLabel{$L \vee$}
			\BIC{$ \Gamma, \nabla^n (B \vee C), \nabla^n (B \vee C) \Rightarrow \Delta$}		
		\end{prooftree}
		By Lemma \ref{lem:inv} for $\D_0$ and $\D_1$, we have proofs for $\Gamma, \nabla^n B, \nabla^n B \Rightarrow \Delta$ and $\Gamma, \nabla^n C, \nabla^n C \Rightarrow \Delta$ of heights less than $h$. By induction hypothesis, we have proofs for $\Gamma, \nabla^n B \Rightarrow \Delta$ and $\Gamma, \nabla^n C \Rightarrow \Delta$ of heights less than $h$. By $L \vee$ we have the desired proof with height at most $h$.
	
		\item[$(L \rightarrow)$] If the last rule is $L \rightarrow$, then $A = \nabla^{n+1} (B \rightarrow C)$ and the last rule is in the form:
		\begin{prooftree}
			\AXC{$\D_0$} \noLine
			\UIC{$\Gamma, \nabla^{n+1} (B \rightarrow C), \nabla^{n+1} (B \rightarrow C) \Rightarrow \nabla^n B$}
			\AXC{$\D_1$} \noLine
			\UIC{$\Gamma, \nabla^n C, (\nabla^{n+1} (B \rightarrow C)), \nabla^{n+1} (B \rightarrow C) \Rightarrow \Delta$}
			\RightLabel{$L \rightarrow$}
			\BIC{$\Gamma, \nabla^{n+1} (B \rightarrow C), \nabla^{n+1} (B \rightarrow C) \Rightarrow \Delta$}
		\end{prooftree}
		By induction hypothesis, we have a proof for $\Gamma, \nabla^{n+1} (B \rightarrow C) \Rightarrow \nabla^n B$ and $\Gamma, \nabla^n C, \nabla^{n+1} (B \rightarrow C) \Rightarrow \Delta$ with height less than $h$. Finally, by $L \rightarrow$ we have the desired proof with height at most $h$.

	
		\item[$(L \supset)$] If the last rule is $L \supset$, then $A = \nabla^n (B \supset C)$ and the last rule is in the form:
		\begin{prooftree}
			\AXC{$\D_0$} \noLine
			\UIC{$\Gamma, \nabla^n (B \supset C), \nabla^n (B \supset C) \Rightarrow \nabla^n B$}
			\AXC{$\D_1$} \noLine
			\UIC{$\Gamma, \nabla^n C, \nabla^n (B \supset C) \Rightarrow \Delta$}
			\RightLabel{$L \supset$}
			\BIC{$\Gamma, \nabla^n (B \supset C), \nabla^n (B \supset C) \Rightarrow \Delta$}
		\end{prooftree}
		By Lemma \ref{lem:inv} for $\D_1$, we have a proof for $\Gamma, \nabla^n C, \nabla^n C \Rightarrow \Delta$ of height less than $h$. By induction hypothesis, we have proofs for $\Gamma, \nabla^n (B \supset C) \Rightarrow \nabla^n B$ and $\Gamma, \nabla^n C \Rightarrow \Delta$, both with heights less than $h$. Finally, by $L \supset$ we have the desired proof, with height at most $h$.
	\end{itemize}
\end{proof}

It remains to show that $cut$ is admissible in $\LDL$.

\begin{thm}\label{thm:cut-adm}
  If $\LDL \vdash \Gamma \Rightarrow A$ and $\LDL \vdash \Sigma, A \Rightarrow \Delta$, then $\LDL \vdash \Gamma, \Sigma \Rightarrow \Delta$.
\end{thm}

We postpone the proof of Theorem \ref{thm:cut-adm} to the next section. Nevertheless, we will use its result to show that $\LDL$ and $\GSTN$ are equivalent, and therefore, share the same logic $\ldl$.

\begin{nota}
  Thanks to Theorems \ref{thm:id-adm}, \ref{thm:lc-adm} and \ref{thm:cut-adm}, from now on, we pretend that the rules $Id$, $Lc$ and $cut$ are available in $\LDL$ as well. Therefore, we use them in the proof-trees in $\LDL$ without any further explanation.
\end{nota}

Now, we are ready to prove the equivalence of two systems.

\begin{thm}\label{thm:ldl-eq-gstn}
  Any sequent $\Gamma \Rightarrow \Delta$ in the language $\L$ is provable in $\GSTN$ iff it is provable in $\LDL$.
\end{thm}
\begin{proof}
  By induction on the proof-tree for $\Gamma \Rightarrow \Delta$, the claim is easy to observe in both directions. For the left to right direction, as was observed in Remark \ref{rem:ldl-gstn}, it suffices to use Theorems \ref{thm:id-adm}, \ref{thm:lc-adm} and \ref{thm:cut-adm}. For the other direction, use Lemma \ref{lem:l-nabla-dist}. We will explain the case where the $\LDL$ proof-tree for $\Gamma \Rightarrow \Delta$ ends with $L \rightarrow ^n$. In this case, by inductin hypothesis, we have proof-trees for $\Gamma, \nabla^{n+1} (A \rightarrow B) \Rightarrow \nabla^n A$ and $\Gamma, \nabla^{n+1} (A \rightarrow B), \nabla^n B \Rightarrow \Delta$ in $\GSTN$. We also have $\nabla^n (A \rightarrow B) \Rightarrow \nabla^n A \rightarrow \nabla^n B$ from Lemma \ref{lem:l-nabla-dist} part (3).

  \begin{prooftree}
    \AXC{$\nabla^n (A \rightarrow B) \Rightarrow \nabla^n A \rightarrow \nabla^n B$}
    \RightLabel{$N$}
    \UIC{$\nabla^{n+1} (A \rightarrow B) \Rightarrow \nabla (\nabla^n A \rightarrow \nabla^n B)$}

    \AXC{$\Gamma, \nabla^{n+1} (A \rightarrow B) \Rightarrow \nabla^n A$}
    \AXC{$\Gamma, \nabla^{n+1} (A \rightarrow B), \nabla^n B \Rightarrow \Delta$}
    \RightLabel{$R \rightarrow$}
    \BIC{$\Gamma, \nabla^{n+1} (A \rightarrow B), \nabla (\nabla^n A \rightarrow \nabla^n B) \Rightarrow \Delta$}

    \RightLabel{$cut$}
    \BIC{$\Gamma, (\nabla^{n+1} (A \rightarrow B))^2 \Rightarrow \Delta$}
    \RightLabel{$Lc$}
    \UIC{$\Gamma, \nabla^{n+1} (A \rightarrow B) \Rightarrow \Delta$}
  \end{prooftree}

  Other cases are similar.
\end{proof}

\begin{cor}
  The logic of dynamic locales $\ldl$ is the logic of the system $\LDL$.
\end{cor}