\begin{thm}[Height-preserving admissibility of $Lc$]\label{thm:lc-adm} For any sequent $\Gamma$, $\Delta$ and $A$, and any $h \geq 0$, if $\LDL \vdash_h \Gamma, A, A \Rightarrow \Delta$ then $\LDL \vdash_h \Gamma, A \Rightarrow \Delta$.
\end{thm}
\begin{proof}
  By induction on $h$. For $h = 0$, the proof for $\Gamma, A, A \Rightarrow \Delta$ must be an axiom, which is impossible. For $h > 0$, we investigate different cases by the last rule of the proof of $\Gamma, A, A \Rightarrow \Delta$. The cases where none of occurrences of $A$ are principal are trivial, as it is enough to apply the same rule on the proofs the induction hypothesis provides.
  
  If one occurrence of $A$ is principal, we have the following cases based on the last rule of the proof:
	\begin{itemize}
		\item[$(L \wedge)$] If the last rule is $L \wedge$, then $A = \nabla^n (B \wedge C)$ it is in the form:

		\begin{prooftree}
			\AXC{$\D_0$} \noLine
			\UIC{$\Gamma, \nabla^n B, \nabla^n C, \nabla^n (B \wedge C) \Rightarrow \Delta$}
			\RightLabel{$L \wedge$}
			\UIC{$\Gamma, \nabla^n (B \wedge C), \nabla^n (B \wedge C) \Rightarrow \Delta$}		
		\end{prooftree}

		Then, by Lemma \ref{lem:inv} we have a proof for $\Gamma, \nabla^n B, \nabla^n B, \nabla^n C, \nabla^n C \Rightarrow \Delta$ of height less than $h$. Using induction hypothesis twice, we would have a proof for $\Gamma, \nabla^n B, \nabla^n C \Rightarrow \Delta$ of height less than $h$. Finally, $L \wedge$ will result in the desired proof with height at most $h$.
	
		\item[$(L \vee)$] If the last rule is $L \vee$, then $A = \nabla^n (B \vee C)$ and the last rule is in the form:
		\begin{prooftree}
			\AXC{$\D_0$} \noLine
			\UIC{$ \Gamma, \nabla^n B, \nabla^n (B \vee C) \Rightarrow \Delta$}
			\AXC{$\D_1$} \noLine
			\UIC{$\Gamma, \nabla^n C, \nabla^n (B \vee C) \Rightarrow \Delta$}
			\RightLabel{$L \vee$}
			\BIC{$ \Gamma, \nabla^n (B \vee C), \nabla^n (B \vee C) \Rightarrow \Delta$}		
		\end{prooftree}
		By Lemma \ref{lem:inv} for $\D_0$ and $\D_1$, we have proofs for $\Gamma, \nabla^n B, \nabla^n B \Rightarrow \Delta$ and $\Gamma, \nabla^n C, \nabla^n C \Rightarrow \Delta$ of heights less than $h$. By induction hypothesis, we have proofs for $\Gamma, \nabla^n B \Rightarrow \Delta$ and $\Gamma, \nabla^n C \Rightarrow \Delta$ of heights less than $h$. By $L \vee$ we have the desired proof with height at most $h$.
	
		\item[$(L \rightarrow)$] If the last rule is $L \rightarrow$, then $A = \nabla^{n+1} (B \rightarrow C)$ and the last rule is in the form:
		\begin{prooftree}
			\AXC{$\D_0$} \noLine
			\UIC{$\Gamma, \nabla^{n+1} (B \rightarrow C), \nabla^{n+1} (B \rightarrow C) \Rightarrow \nabla^n B$}
			\AXC{$\D_1$} \noLine
			\UIC{$\Gamma, \nabla^n C, (\nabla^{n+1} (B \rightarrow C)), \nabla^{n+1} (B \rightarrow C) \Rightarrow \Delta$}
			\RightLabel{$L \rightarrow$}
			\BIC{$\Gamma, \nabla^{n+1} (B \rightarrow C), \nabla^{n+1} (B \rightarrow C) \Rightarrow \Delta$}
		\end{prooftree}
		By induction hypothesis, we have a proof for $\Gamma, \nabla^{n+1} (B \rightarrow C) \Rightarrow \nabla^n B$ and $\Gamma, \nabla^n C, \nabla^{n+1} (B \rightarrow C) \Rightarrow \Delta$ with height less than $h$. Finally, by $L \rightarrow$ we have the desired proof with height at most $h$.

	
		\item[$(L \supset)$] If the last rule is $L \supset$, then $A = \nabla^n (B \supset C)$ and the last rule is in the form:
		\begin{prooftree}
			\AXC{$\D_0$} \noLine
			\UIC{$\Gamma, \nabla^n (B \supset C), \nabla^n (B \supset C) \Rightarrow \nabla^n B$}
			\AXC{$\D_1$} \noLine
			\UIC{$\Gamma, \nabla^n C, \nabla^n (B \supset C) \Rightarrow \Delta$}
			\RightLabel{$L \supset$}
			\BIC{$\Gamma, \nabla^n (B \supset C), \nabla^n (B \supset C) \Rightarrow \Delta$}
		\end{prooftree}
		By Lemma \ref{lem:inv} for $\D_1$, we have a proof for $\Gamma, \nabla^n C, \nabla^n C \Rightarrow \Delta$ of height less than $h$. By induction hypothesis, we have proofs for $\Gamma, \nabla^n (B \supset C) \Rightarrow \nabla^n B$ and $\Gamma, \nabla^n C \Rightarrow \Delta$, both with heights less than $h$. Finally, by $L \supset$ we have the desired proof, with height at most $h$.
	\end{itemize}
\end{proof}