
\begin{lem}[Inversion]\label{lem:inv} \quad
	\begin{enumerate}
		\item If $\LDL \vdash_h \Gamma, \nabla^n (A \wedge B) \Rightarrow \Delta$ then $\LDL \vdash_h \Gamma, \nabla^n A, \nabla^n B \Rightarrow \Delta$.
		\item If $\LDL \vdash_h \Gamma, \nabla^n (A \vee B) \Rightarrow \Delta$ then $\LDL \vdash_h \Gamma, \nabla^n A \Rightarrow \Delta$ and $\LDL \vdash_h \Gamma, \nabla^n B \Rightarrow \Delta$.
  	\item If $\LDL \vdash_h \Gamma, \nabla^n (A \supset B) \Rightarrow \Delta$ then $\LDL \vdash_h \Gamma, \nabla^n B \Rightarrow \Delta$.
	\end{enumerate}
\end{lem}
\begin{proof}
  We only explain the proof for (3), the other two parts are similar. We use induction on $h$, the height of the proof-tree for $\Gamma, \nabla^n (A \supset B) \Rightarrow \Delta$. It is clear that $h$ can not be $0$, since $\Gamma, \nabla^n (A \supset B) \Rightarrow \Delta$ can not be proved by an axiom. If the proof-tree ends with $L \supset ^n$, and $\nabla^n (A \supset B)$ is the principal formula, then the claim would be obvious. In any other case, just commute the last rule with the sequent obtained from the induction hypothesis. Here we explain the case for $N$.

  Suppose the last rule is $N$. So we have
  \begin{prooftree}
    \AXC{$\Gamma, \nabla^n (A \supset B) \Rightarrow \Delta$} \RightLabel{$N$}
    \UIC{$\nabla \Gamma, \nabla^{n+1} (A \supset B) \Rightarrow \nabla \Delta$}
  \end{prooftree}
  at the end of a proof-tree of height $h > 0$. From the induction hypothesis for the premise, we have a proof for $\Gamma, \nabla^n B \Rightarrow \Delta$ of heigt $h - 1$. Using $N$, we have a proof-tree for $\nabla \Gamma, \nabla^{n+1} B \Rightarrow \nabla \Delta$ with height $h$.

  The cases for the other rules are similar.
\end{proof}